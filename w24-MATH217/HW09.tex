\documentclass[12pt]{exam}
\usepackage{amsmath}
\usepackage{amssymb}
\usepackage{amsthm}
\usepackage{tikz}
\usepackage{mathtools}
\usepackage{graphicx}
\usepackage{wrapfig}

\usepackage{bm} %bold symbols
\usepackage{hyperref} %add links

%%%%%%%%%%%%%%%%%%%%%%%%%
% 	Define vars here 	%
%%%%%%%%%%%%%%%%%%%%%%%%%

\def\hwName{Homework Set Part B}
\author{Zhengyu James Pan} %use like \@author
\def\instructor{Dr. Paul Kessenich}
\def\email{jzpan@umich.edu}
\def\dueDate{SUNDAY, MARCH 31}

\newcommand{\reals}{\mathbb{R}}
\newcommand{\realsn}{\reals^{n}}
\newcommand{\nxn}{n\times n}
\newcommand{\realsnxn}{\reals^{\nxn}}
\newcommand{\naturals}{\mathbb{N}}
\newcommand{\ints}{\mathbb{Z}}
\newcommand{\transpose}{^\top}

\makeatletter

\begin{document}
%Header
\pagestyle{head}
\firstpageheader{}{}{}
\header{MATH 217}{\hwName}{\thepage}

%Solution formatting
\printanswers
\unframedsolutions

%Top matter
{\parindent0in
\bf
\begin{center}
	MATH 217 W24 - LINEAR ALGEBRA, Section 001 ({\instructor}) \\
	{\hwName} due {\dueDate} at 11:59pm \\
	\@author\ (\href{mailto:\email}{\email})
\end{center}
}

\begin{questions}
\question Consider the four points (2, 4, 6), (1, 3, 2), (1, 1, 0) and (1, 2, 3) in $\reals^3$.
	\begin{parts}
		\part Write a matrix equation that, if it were consistent, could be used to find the coefficients $A, B, C$ in the equation of a plane of the form $z = Ax + By + C$ that contains all four
        points.
		\begin{solution}
			\[ \begin{bmatrix}
                2 & 4 & 1 \\
                1 & 3 & 1 \\
                1 & 1 & 1 \\
                1 & 2 & 1 \\
            \end{bmatrix} \vec x = \begin{bmatrix}
                6 \\ 2 \\ 0 \\ 3
            \end{bmatrix} \]
		\end{solution}
        \part Show that the matrix equation from (a) is, in fact, inconsistent.
            \begin{solution}
                We row reduce to show the augmented matrix is inconsistent.
                \begin{align*}
                    \left[\hspace{-5pt}\begin{array}{ccc|c}
                        2 & 4 & 1 & 6 \\
                        1 & 3 & 1 & 2 \\
                        1 & 1 & 1 & 0 \\
                        1 & 2 & 1 & 3 \\
                      \end{array}\hspace{-5pt}\right] &= \left[\hspace{-5pt}\begin{array}{ccc|c}
                        2 & 4 & 1 & 6 \\
                        0 & 2 & 0 & 2 \\
                        1 & 1 & 1 & 0 \\
                        0 & 1 & 0 & 3 \\
                      \end{array}\hspace{-5pt}\right] \\
                      &= \left[\hspace{-5pt}\begin{array}{ccc|c}
                        2 & 4 & 1 & 6 \\
                        0 & 1 & 0 & 1 \\
                        1 & 1 & 1 & 0 \\
                        0 & 1 & 0 & 3 \\
                      \end{array}\hspace{-5pt}\right] \\
                      &= \left[\hspace{-5pt}\begin{array}{ccc|c}
                        2 & 4 & 1 & 6 \\
                        0 & 1 & 0 & 1 \\
                        1 & 1 & 1 & 0 \\
                        0 & 0 & 0 & 2 \\
                      \end{array}\hspace{-5pt}\right]
                \end{align*}
            \end{solution}
        \part Now write a matrix equation that can be used to find the least-squares solution to the equation you wrote in (a). Fully simplify any matrix products that occur in your equation, but do not (yet) attempt to solve the equation.
            \begin{solution}
                The normal equation is
                $A\transpose A = A\transpose B$.
                \begin{align*}
                    \begin{bmatrix}
                        2 & 1 & 1 & 1\\
                        4 & 3 & 1 & 2\\
                        1 & 1 & 1 & 1\\
                    \end{bmatrix}\begin{bmatrix}
                        2 & 4 & 1 \\
                        1 & 3 & 1 \\
                        1 & 1 & 1 \\
                        1 & 2 & 1 \\
                    \end{bmatrix}\vec x &= \begin{bmatrix}
                        2 & 1 & 1 & 1\\
                        4 & 3 & 1 & 2\\
                        1 & 1 & 1 & 1\\
                    \end{bmatrix} \begin{bmatrix}
                        6 \\ 2 \\ 0 \\ 3
                    \end{bmatrix} \\
                    \begin{bmatrix}
                        7 & 14 & 5\\
                        14 & 30 & 10\\
                        5 & 10 & 4\\
                    \end{bmatrix} \vec x &= \begin{bmatrix}
                        17 \\ 36 \\ 11
                    \end{bmatrix}
                \end{align*}
            \end{solution}
        \part Now, solve your equation using methods taught in this course. (You can use a matrix calculator to check your answer, but you must be able to solve this problem by hand.)
            \begin{solution}
                \begin{align*}
                    \begin{bmatrix}
                        7 & 14 & 5\\
                        14 & 30 & 10\\
                        5 & 10 & 4\\
                    \end{bmatrix} \vec x &= \begin{bmatrix}
                        17 \\ 36 \\ 11
                    \end{bmatrix} \\
                    \left[\hspace{-5pt}\begin{array}{ccc|c}
                        7 & 14 & 5 & 17 \\
                        14 & 30 & 10 & 36 \\
                        5 & 10 & 4 & 11 \\
                      \end{array}\hspace{-5pt}\right] &= \left[\hspace{-5pt}\begin{array}{ccc|c}
                        7 & 14 & 5 & 17 \\
                        0 & 2 & 0 & 2 \\
                        5 & 10 & 4 & 11 \\
                      \end{array}\hspace{-5pt}\right] \\
                      &= \left[\hspace{-5pt}\begin{array}{ccc|c}
                        2 & 0 & 1 & 2 \\
                        0 & 1 & 0 & 1 \\
                        5 & 0 & 4 & 1 \\
                      \end{array}\hspace{-5pt}\right] \\
                      &= \left[\hspace{-5pt}\begin{array}{ccc|c}
                        2 & 0 & 1 & 2 \\
                        0 & 1 & 0 & 1 \\
                        0 & 0 & 3/2 & -4 \\
                      \end{array}\hspace{-5pt}\right] \\
                      &= \left[\hspace{-5pt}\begin{array}{ccc|c}
                        2 & 0 & 1 & 2 \\
                        0 & 1 & 0 & 1 \\
                        0 & 0 & 1 & -8/3 \\
                      \end{array}\hspace{-5pt}\right] \\
                      &= \left[\hspace{-5pt}\begin{array}{ccc|c}
                        2 & 0 & 0 & 14/3 \\
                        0 & 1 & 0 & 1 \\
                        0 & 0 & 1 & -8/3 \\
                      \end{array}\hspace{-5pt}\right] \\
                      &= \left[\hspace{-5pt}\begin{array}{ccc|c}
                        1 & 0 & 0 & 7/3 \\
                        0 & 1 & 0 & 1 \\
                        0 & 0 & 1 & -8/3 \\
                      \end{array}\hspace{-5pt}\right] \\
                      \vec x^* &= \begin{bmatrix}
                        7/3 \\ 1 \\ -8/3
                      \end{bmatrix}
                \end{align*}
            \end{solution}
	\end{parts}
\clearpage

%2
\question \begin{parts}
    \part Which of the following is an inner product in $\mathcal P^2$? Explain.
        \begin{subparts}
            \subpart $\langle f, g \rangle = f(1)g(2) + f (2)g(1) + f (3)g(3)$
            \subpart $\langle f, g \rangle = f (1)g(1) + f (2)g(2) + f (3)g(3)$
        \end{subparts}
        \begin{solution}
            Inner product (ii) is a valid inner product in $\mathcal P^2$: it is symmetric, bilinear (by distrubitivity), and positive definite (since squares of nonzero reals are always positive). 
            \par However, (i) is not an inner product, since it is not always positive definite. For instance, $f(x) = 2x^{2}-8x+7$ has $f(1) = 1$, $f(2) = -1$, $f(3) = 1$. So $\langle f, f \rangle_{(i)} = -1 - 1 + 1 = -1$.
        \end{solution}
    \part Let $V = C^\infty[-1, 1]$, the vector space of smooth functions on the interval $[-1, 1]$. Which of the following is an inner product in $V$? Explain.
        \begin{subparts}
            \subpart $\langle f, g \rangle = \int_{-1}^{1}xf(x)g(x)dx$
            \subpart $\langle f, g \rangle = \int_{-1}^{1}x^2f(x)g(x)dx$
        \end{subparts}
    \begin{solution}
        Function (i) is not an inner product: Consider $f(x) = 1$. Then $\langle f, f \rangle_(i) = \int_{-1}^{1}x\, dx = 0$. But $f$ is not 0. So (i) is not positive definite.
        \par Inner product (ii) is a valid inner product. By commutativity of multiplication, it is symmetric. By distributivity of multiplication and linearity of integrals, \[\langle f + h, g \rangle_{(ii)} = \int_{-1}^{1}x^2 (f+h)g \, dx = \int_{-1}^{1}x^2 (fg+hg) \, dx = \int_{-1}^{1}x^2 fg\, dx + \int_{-1}^{1} x^2hg \, dx \, dx.\]
        Finally, since $x^2$ is always nonnegative and $f(x)^2$ is positive definite, $\langle f, f \rangle$ will be positive definite. Note that if non-smooth functions were in this vector space, one could set $f(0) = 1, f(x \neq 0) = 0$ and this inner product would not be positive definite. However, since the vector space is smooth functions, something like this is not possible.
    \end{solution}
\end{parts}
\clearpage

\question Let $V = C^\infty \left[ -\frac{\pi}{2}, \frac{\pi}{2}\right]$, the vector space of smooth functions on the interval $\left[ -\frac{\pi}{2}, \frac{\pi}{2}\right]$, and consider the inner product defined by $\langle f, g\rangle = \int_{-\pi/2}^{\pi/2} f (x)g(x) \sin^2(x) dx$. (You do not need to show that this is an inner product, but make sure that you would be able to do so if it were an exam question!) Let $W = $ span$(1, x, x^2)$.
\par In what follows, you may feel free to use an online integral calculator (e.g. Wolfram Alpha) to evaluate any difficult integrals, but make sure that your work shows clearly what integrals you are computing, and how you are making use of the results. Results may be expressed using either exact expressions (e.g., $\pi/\sqrt2$) or decimal approximations (e.g., 2.2214), but if you use decimal approximations, please retain at least four digits' worth of precision.
    \begin{parts}
        \part Compute each of the following.
            \begin{subparts}
                \subpart $\langle 1, x \rangle $
                    \begin{solution}
                        \begin{align*}
                            \langle 1, x\rangle &= \int_{-\pi/2}^{\pi/2} 1 \cdot x \sin^2(x) dx \\
                            &= \int_{-\pi/2}^{\pi/2} x \sin^2(x) dx \\
                            &= \left[ -\dfrac{2x\sin\left(2x\right)+\cos\left(2x\right)-2x^2}{8} \right]_{-\pi/2}^{\pi/2} \tag{calculator} \\
                            &= 0 \tag{calculator} \\
                        \end{align*}
                    \end{solution}
                \subpart $||1||$
                    \begin{solution}
                        \begin{align*}
                            \langle 1, 1\rangle &= \int_{-\pi/2}^{\pi/2} 1 \sin^2(x) dx \\
                            &= \int_{-\pi/2}^{\pi/2} \sin^2(x) dx \\
                            &= \frac{\pi}{2} \tag{calculator}\\
                        \end{align*}
                    \end{solution}
                \subpart $||x||$
                    \begin{solution}
                        \begin{align*}
                            \langle x, x\rangle &= \int_{-\pi/2}^{\pi/2} x \cdot x \sin^2(x) dx \\
                            &= \int_{-\pi/2}^{\pi/2} x^2 \sin^2(x) dx \\
                            &= \dfrac{{\pi}^3+6{\pi}}{24} \approx 2.077326358409941 \tag{calculator}\\
                        \end{align*}
                    \end{solution}
            \end{subparts}
        \part Find a basis $\mathcal U$ for the subspace $W$ that is orthonormal relative to the given inner product.
            \begin{solution}
                We utilize the Gram-Schmidt process on the given basis of W, $\{1, x, x^2\}$. We know already that $1$ and $x$ are orthogonal; First, we normalize.
                    \begin{align*}
                        \langle 1, 1 \rangle &= \pi / 2\\
                        u_1 &= \sqrt{\frac{2}{\pi}} \\\\
                        \langle x, x \rangle &= \dfrac{{\pi}^3+6{\pi}}{24} \\
                        u_2 &= \sqrt{\dfrac{24}{{\pi}^3+6{\pi}}}x \approx 0.4813880091x \\
                    \end{align*}
                Then, we find the component of $x^2$ which is orthogonal to both of these.
                \begin{align*}
                    \langle 1, x^2 \rangle &= \int_{-\pi/2}^{\pi/2} x^2 \sin^2(x) dx \\
                    &= \dfrac{{\pi}^3+6{\pi}}{24} \approx 2.077326358409941 \tag{calculator}\\
                    \text{proj}_1 x^2 =& \frac{2\left(\dfrac{{\pi}^3+6{\pi}}{24}\right)}{\pi}\\
                    &= \frac{\pi^2 + 6}{12} \\
                \end{align*}
                \begin{align*}
                    \langle x, x^2 \rangle &= \int_{-\pi/2}^{\pi/2} x^3 \sin^2(x) dx \\
                    &= 0 \tag{odd integrand}\\
                    \text{proj}_x x^2 &= 0
                    \\\\
                    x^{2\perp} = x^2 - \frac{\pi^2 + 6}{12}
                \end{align*}
            \end{solution}
        \part Let $h \in C^\infty \left[ -\frac{\pi}{2}, \frac{\pi}{2}\right]$ be the function defined by $h(x) = e^x$ for all $x \in \left[ -\frac{\pi}{2}, \frac{\pi}{2}\right]$. Compute proj$_W h$.
    \end{parts}
\end{questions}

\end{document}