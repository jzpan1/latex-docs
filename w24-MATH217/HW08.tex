\documentclass[12pt]{exam}
\usepackage{amsmath}
\usepackage{amssymb}
\usepackage{amsthm}
\usepackage{tikz}
\usepackage{mathtools}
\usepackage{graphicx}
\usepackage{wrapfig}

\usepackage{bm} %bold symbols
\usepackage{hyperref} %add links

%%%%%%%%%%%%%%%%%%%%%%%%%
% 	Define vars here 	%
%%%%%%%%%%%%%%%%%%%%%%%%%

\def\hwName{Homework Set Part B}
\author{Zhengyu James Pan} %use like \@author
\def\instructor{Dr. Paul Kessenich}
\def\email{jzpan@umich.edu}
\def\dueDate{???}

\newcommand{\reals}{\mathbb{R}}
\newcommand{\realsn}{\reals^{n}}
\newcommand{\nxn}{n\times n}
\newcommand{\realsnxn}{\reals^{\nxn}}
\newcommand{\naturals}{\mathbb{N}}
\newcommand{\ints}{\mathbb{Z}}
\newcommand{\transpose}{^\top}

\makeatletter

\begin{document}
%Header
\pagestyle{head}
\firstpageheader{}{}{}
\header{MATH 217}{\hwName}{\thepage}

%Solution formatting
\printanswers
\unframedsolutions

%Top matter
{\parindent0in
\bf
\begin{center}
	MATH 217 W24 - LINEAR ALGEBRA, Section 001 ({\instructor}) \\
	{\hwName} due {\dueDate} at 11:59pm \\
	\@author\ (\href{mailto:\email}{\email})
\end{center}
}

\begin{questions}

%1
\question Let $W$ be a subspace of $\reals^n$ and let $\mathcal B = (\vec v_1, . . . , \vec v_d)$ be a basis for $W$. Consider the transformation $\reals^n \xrightarrow{\pi} \reals^n$ defined by
\[\pi(\vec v) = \sum_{i=1}^d \frac{\vec v \cdot \vec v_i}{\vec v_i \cdot \vec v_i}\vec v_i.\]
	\begin{parts}
		\part Show that if $\vec v_i \cdot \vec v_j = 0$ for all $1 \leq i \neq j \leq d$, then the transformation $\pi$ is the orthogonal projection onto $W$. (Note: this is almost, but not quite, the way we defined orthogonal projection. Make sure you understand how our definition is different from this before you start trying to prove it!)
			\begin{solution}
			\end{solution}
		\part Give a counterexample to show that if the basis vectors in $\mathcal B$ are not perpendicular to each other, then the linear transformation $\pi$ defined above $\pi$ is not orthogonal projection onto $W$.
			\begin{solution}
				\textbf{HELPER DEFINITION: Orthogonal Projection.} If $W$ is a subspace of an inner product space $V$ and if $\vec v \in V$, the orthogonal projection of $\vec v$ onto $W$ is the unique vector $\vec w \in W$ such that $\vec v - \vec w \in W^{\perp}$. The orthogonal projection of $\vec v$ onto $W$ is sometimes denoted proj$_W (\vec v)$.
			\end{solution}
	\end{parts}
\clearpage

%2
\question Let $\mathcal O_n \subseteq \realsnxn$ denote the set of orthogonal $\nxn$ matrices. Determine whether each of the following statements is True or False, and provide a short proof (or a counter-example) of your claim.
	\begin{parts}
		\part $\mathcal O_n$ is a subspace of $\realsnxn$.
		\part If $A, B \in \mathcal O_n$, then $AB \in \mathcal O_n$.
		\part If $A \in \mathcal O_n$, then $A^2 \in \mathcal O_n$.
		\part If $A^2 \in \mathcal O_n$, then $A \in \mathcal O_n$.
		\part If $A \in \mathcal O_n$ and $A^2$ is the identity matrix, then $A$ is symmetric.
	\end{parts} 
\clearpage

%3
\question \begin{parts}
	\part Suppose that $\mathcal B = (\vec b_1, . . . , \vec b_r)$ is an orthonormal basis of a subspace $V$ of $\realsn$.
	Prove that for all $\vec v, \vec w \in V, [\vec v]_\mathcal B \cdot [\vec w]_\mathcal B = \vec v \cdot \vec w$
	\part Prove that if $\mathcal B = (\vec b_1, . . . , \vec b_r)$ and $\mathcal C = (\vec c_1, . . . , \vec c_r)$ are two orthonormal bases of $V $, then
	$S_{\mathcal B \to \mathcal C}$ is an orthogonal $r \times r$ matrix.
\end{parts}
\end{questions}

\end{document}