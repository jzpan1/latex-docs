\documentclass[12pt]{exam}
\usepackage{amsmath}
\usepackage{amssymb}
\usepackage{amsthm}
\usepackage{tikz}
\usepackage{mathtools}
\usepackage{graphicx}
\usepackage{wrapfig}

\usepackage{bm} %bold symbols
\usepackage{hyperref} %add links

%%%%%%%%%%%%%%%%%%%%%%%%%%%%%%%%%
% 	Define vars/macros here 	%
%%%%%%%%%%%%%%%%%%%%%%%%%%%%%%%%%

\def\hwName{Homework Set Part B}
\author{Zhengyu James Pan} %use like \@author
\def\instructor{Dr. Paul Kessenich}
\def\email{jzpan@umich.edu}
\def\dueDate{Thursday, January 25}

\newcommand{\reals}{\mathbb{R}}
\newcommand{\naturals}{\mathbb{N}}


\makeatletter

\begin{document}
%Header
\pagestyle{head}
\firstpageheader{}{}{}
\header{MATH 217}{\hwName}{\thepage}

%Solution formatting
\printanswers
\unframedsolutions

%Top matter
{\parindent0in
\bf
\begin{center}
	MATH 217 W24 - LINEAR ALGEBRA, Section 001 ({\instructor}) \\
	{\hwName} due {\dueDate} at 11:59pm \\
	\@author\ (\href{mailto:\email}{\email})
\end{center}
}

\begin{questions}
\question In parts (a) - (d) below, determine whether the given function is injective, surjective,
both, or neither. Justify your answers.
	\begin{parts}
		\part the function $f : [0, 4] \rightarrow [0, 18]$ defined by $f (x) = x^2 + 2$;
		\begin{solution}
			Injective. If $f(x_1) = f(x_2)$, it follows that $x_1^2 = x_2^2$. Since the domain is positive, this also means $x_1 = x_2$, showing injectivity. There is no solution in the domain to $f(x) = 0$, so there exists a value in the codomain which is not in the image of $f$. Thus, the function is not surjective.
		\end{solution}
        \part the function $g : \mathbb{R} \rightarrow \mathbb{R}$ defined by $g(x) = 2x - 5$;
            \begin{solution}
                Bijective. If $g(x_1) = g(x_2)$, $2x_1 - 5 = 2x_2 - 5$. Therefore, $x_1 = x_2$, showing injectivity. Let $y\in\mathbb{R}$, and $x = \frac{y+5}{2}$. Then $x\in\reals$, and $g(x) = y$. Thus $g$ is surjective.
            \end{solution}
        \part the function $h : \mathbb{R}^2 \rightarrow \mathbb{R}$ defined by $h(x, y) = 2x^2 + 5y^2$;
            \begin{solution}
                Neither. $10 = h\left(\sqrt{5}, 0\right) = h\left(0, \sqrt{2}\right)$, so $h$ is not injective. $h(x, y) = -2$ has no solutions in $\reals^2$ since a square cannot be a negative number, therefore $h$ is not surjective.
            \end{solution}
        \part the function $q : \naturals \rightarrow \naturals$ defined by $q(n) = \begin{cases*}
            n, &\text{if n is odd} \\
            n/2 &\text{if n is even.}
        \end{cases*}$
            \begin{solution}
                Surjective. $1 = q(1) = q(2)$, so $q$ is not injective. Let $m \in \naturals, n = 2m$. Then $n \in \naturals$, $n$ is even, and $q(n) = m$. So $q$ is surjective.
            \end{solution}
	\end{parts}

%2
\question Determine whether each statement is true or false. If it is true, prove it. If it is false, prove this by giving a counterexample.
    \begin{parts}
        \part For every function $f : X \rightarrow Y$ and all $A, B \subseteq X$, if $A \cap B = \emptyset$, then $f [A] \cap f [B] = \emptyset$.
            \begin{solution}
                False. Let $f : \reals \rightarrow \reals$ be defined by $f(x) = x^2$. Assign $A = \reals^+$, $B = \reals^-$. Then $A \cap B = \emptyset$, but $f(1\in A) = f(-1 \in B) = 1$. Therefore $f[A] \cap f[B] \neq \emptyset$. 
            \end{solution}
        \part For every function $f : X \rightarrow Y$ and all $A, B \subseteq X$, if $f [A] \cap f [B] = \emptyset$, then $A \cap B = \emptyset$.
            \begin{solution}
                True. We prove the contrapositive. Take any $f : X \rightarrow Y$ and $A, B \subseteq X$ such that $A \cap B$. Then $\exists a \in X$ such that $a \in A \cap B$. Since $a\in A, f(a) \in f[A]$. Similarly, $a \in B,$ so $f(a) \in f[B]$. As $f(a) \in f[A]$ and $f(a) \in f[B]$, $f(a) \in f[A] \cap f[B]$. This means that for every function $f : X \rightarrow Y$ and all $A, B \subseteq X$, if $A \cap B \neq \emptyset$, then $f [A] \cap f [B] \neq \emptyset$. Thus the contrapositive is true, so the original statement is true.
            \end{solution}
        \part For every function $f : X \rightarrow Y$ and all $A \subseteq X$, we have $f^{-1}[f [A]] = A$.
            \begin{solution}
                False. Let $f : \reals \rightarrow \reals$ be defined by $f(x) = x^2$. Assign $A = \{1\}.$ Then $f[A] = \{1\}$. However, $f(1) = f(-1) = 1$, so $f^{-1}[f[A]] = \{-1, 1\} \neq A$.
            \end{solution}
        \part For every function $f : X \rightarrow Y$ and all $A \subseteq X$, we have $f [X \backslash A] = Y \backslash f [A]$.
            \begin{solution}
                False. Let $f : \reals \rightarrow \reals$ be defined by $f(x) = x^2$. Assign $A = \{1\}$. Then $f[A] = \{1\}$, but $f(1) = f(-1) = 1$. Therefore $f[A] \subseteq f[X\backslash A]$, and $f[X\backslash A] \neq Y\backslash f[A]$. 
            \end{solution}
        \part For every bijective function $f : X \rightarrow Y$ and all $A, B \subseteq X$, we have $f [A \cap B] = f [A] \cap f [B]$.
            \begin{solution}
                True. \\\\
                Let $x \in f[A \cap B]$. Then let $a = f^{-1}(x)$. We know $a \in A \cap B$ is unique due to bijectivity. Since $a \in A \cap B, a \in A \wedge a \in B$. Thus $f[A \cap B] \subseteq f[A] \cap f[B]$. \\
                Let $y \in f[A] \cap f[B]$. Then let $b = f^{-1}(y)$. We know $b$ is unique due to bijectivity. Additionally, $b \in A \wedge b \in B$ because $y \in f[A] \cap f[B]$. Since $b \in A \cap B, f(b) = y \in A \wedge a \in B$. Thus $f[A] \cap f[B] \subseteq f[A \cap B]$.\\
                Both $f[A] \cap f[B] \subseteq f[A \cap B]$ and $f[A \cap B] \subseteq f[A] \cap f[B]$, so $f[A \cap B] = f[A] \cap f[B]$.
            \end{solution}
    \end{parts}
    

%3
\question
    \begin{parts}
        \part Prove that for every function $f : \reals \rightarrow \reals$, if $f (cx) = cf (x)$ for all $c \in \reals$ and $x \in \reals$, then $f (x + y) = f (x) + f (y)$ for all $x, y \in \reals$. (In other words, prove that every function $f : \reals \rightarrow \reals$ that preserves scalar multiplication is a linear transformation from $\reals$ to $\reals$.)
            \begin{solution}
                Let $x \neq 0 \in \reals$, $y \in \reals$. Then $y = cx$, where $c\in \reals$, by closure of nonzero division in the reals. Additionally, $f(y) = f(cx) = cf(x)$. Thus $f(x+y) = f(x + cx) = f((1 + c)x) = (1+c)f(x) = f(x) + cf(x) = f(x) + f(y)$. NEEDS MORE
            \end{solution}
        \part Give an example to show that the argument you gave in part (a) cannot work in 2 dimensions. That is, explicitly describe a function $f : \reals^2 \rightarrow \reals^2$ that is not a linear transformation but has the property that $f (c\vec{x}) = cf (\vec{x})$ for all $\vec{x} \in \reals^2$ and $c \in \reals$. Remember to prove that your example works!
    \end{parts}

% %4
% \question Let f : R → R be a function, and suppose that f (x + y) = f (x) + f (y) for all x, y ∈ R. (In other words, suppose that f preserves addition).
%     \begin{parts}
%         \part Prove that f (0) = 0.
%         \part Prove that for all x ∈ R, f (−x) = −f (x).
%         \part Use induction to prove that for all n ∈ N and x ∈ R, f (nx) = nf (x).
%         \part Prove that for all m ∈ Z and x ∈ R, f (mx) = mf (x).
%     \end{parts}

\end{questions}

\end{document}