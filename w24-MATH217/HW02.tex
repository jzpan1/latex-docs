\documentclass[12pt]{exam}
\usepackage{amsmath}
\usepackage{amssymb}
\usepackage{amsthm}
\usepackage{tikz}
\usepackage{mathtools}
\usepackage{graphicx}
\usepackage{wrapfig}

\usepackage{bm} %bold symbols
\usepackage{hyperref} %add links

%%%%%%%%%%%%%%%%%%%%%%%%%
% 	Define vars here 	%
%%%%%%%%%%%%%%%%%%%%%%%%%

\def\hwName{Homework 2 Part B}
\author{Zhengyu James Pan} %use like \@author
\def\email{jzpan@umich.edu}
\makeatletter

\begin{document}
%Header
\pagestyle{head}
\firstpageheader{}{}{}
\header{MATH 217}{\hwName}{\thepage}

%Solution formatting
\printanswers
\unframedsolutions

%Top matter
{\parindent0in
\begin{center}
	\bf MATH 217 - LINEAR ALGEBRA\\
	\bf \hwName, DUE Thursday, January 25 at 11:59pm \\
	\@author\ (\href{mailto:\email}{\email})
\end{center}
}

\begin{questions}
\question In parts (a) – (d) below, determine whether the given function is injective, surjective,
both, or neither. Justify your answers.
	\begin{parts}
		\part the function f : [0, 4] → [0, 18] defined by f (x) = x2 + 2;
		\begin{solution}
			Solution
		\end{solution}
        \part the function g : R → R defined by g(x) = 2x − 5;
        \part the function h : R2 → R defined by h(x, y) = 2x2 + 5y2;
        \part the function q : N → N defined by $q(n) = $\begin{cases*}
            n, if n is odd
        n/2 if n is even.
        \end{cases*}
	\end{parts}

%2
\question Determine whether each statement is true or false. If it is true, prove it. If it is false, prove this by giving a counterexample.
    \part For every function f : X → Y and all A, B ⊆ X, if A ∩ B = ∅, then f [A] ∩ f [B] = ∅.
    \part For every function f : X → Y and all A, B ⊆ X, if f [A] ∩ f [B] = ∅, then A ∩ B = ∅.
    \part For every function f : X → Y and all A ⊆ X, we have f −1[f [A]] = A.
    \part For every function f : X → Y and all A ⊆ X, we have f [X \ A] = Y \ f [A].
    \part For every bijective function f : X → Y and all A, B ⊆ X, we have f [A ∩ B] = f [A] ∩ f [B].

%3
\question
    \begin{parts}
        \part Prove that for every function f : R → R, if f (cx) = cf (x) for all c ∈ R and x ∈ R, then f (x + y) = f (x) + f (y) for all x, y ∈ R. (In other words, prove that every function f : R → R that preserves scalar multiplication is a linear transformation from R to R.)
        \part Give an example to show that the argument you gave in part (a) cannot work in 2 dimen-
        sions. That is, explicitly describe a function f : R2 → R2 that is not a linear transformation
        but has the property that f (c~x) = cf (~x) for all ~x ∈ R2 and c ∈ R. Remember to prove
        that your example works!
    \end{parts}

%4
\question Let f : R → R be a function, and suppose that f (x + y) = f (x) + f (y) for all x, y ∈ R. (In other words, suppose that f preserves addition).
    \begin{parts}
        \part Prove that f (0) = 0.
        \part Prove that for all x ∈ R, f (−x) = −f (x).
        \part Use induction to prove that for all n ∈ N and x ∈ R, f (nx) = nf (x).
        \part Prove that for all m ∈ Z and x ∈ R, f (mx) = mf (x).
    \end{parts}

\end{questions}

\end{document}