\documentclass[12pt]{exam}
\usepackage{amsmath}
\usepackage{amssymb}
\usepackage{amsthm}
\usepackage{tikz}
\usepackage{mathtools}
\usepackage{graphicx}
\usepackage{wrapfig}

\usepackage{bm} %bold symbols
\usepackage{hyperref} %add links

%%%%%%%%%%%%%%%%%%%%%%%%%
% 	Define vars here 	%
%%%%%%%%%%%%%%%%%%%%%%%%%

\def\hwName{Homework Set Part B}
\author{Zhengyu James Pan} %use like \@author
\def\instructor{Dr. Paul Kessenich}
\def\email{jzpan@umich.edu}
\def\dueDate{Thurs, Feb 8}

\newcommand{\reals}{\mathbb{R}}
\newcommand{\naturals}{\mathbb{N}}
\newcommand{\ints}{\mathbb{Z}}
\newcommand{\transpose}{^\top}

\makeatletter

\begin{document}
%Header
\pagestyle{head}
\firstpageheader{}{}{}
\header{MATH 217}{\hwName}{\thepage}

%Solution formatting
\printanswers
\unframedsolutions

%Top matter
{\parindent0in
\bf
\begin{center}
	MATH 217 W24 - LINEAR ALGEBRA, Section 001 ({\instructor}) \\
	{\hwName} due {\dueDate} at 11:59pm \\
	\@author\ (\href{mailto:\email}{\email})
\end{center}
}

\begin{questions}
\question Let $\reals^n \xrightarrow{T} \reals^n$ be a linear transformation. As on HW 3, we define $T^k$ to be the $k$-fold composition of $T$ with itself. Let $A$ be the standard matrix of $T$, by which we mean the unique $n \times n$ matrix such that $T (\vec{x}) = A\vec{x}$ for all $\vec{x} \in \reals^n$.
	\begin{parts}
		\part Prove that for all $k$, the standard matrix for $T^k$ is the matrix $A^k$. [Hint: induction works nicely.]
		\begin{solution}
			We are given that the standard matrix $A^{(1)}$ represents the transformation $T^{(1)}$. Assume that the transformation $T^n$ can be represented by the standard matrix $A^n$. We know by a theorem on the worksheets that the standard matrix of two linear transformations, both from $\reals^n \rightarrow \reals^n$, is equal to the product of their respective standard matrices. Then $(T^n \circ T)(x) = A^n A \vec{x}$. This is equal to $T^{n+1}(x) = A^{n+1}\vec{x}$. So by induction, $T^k (\vec{x}) = A^k\vec{x}$ for all $\vec{x} \in \reals^n$ and $k \in \naturals$.
		\end{solution}
        \part We define $T$ to be nilpotent if there exists some $k \in \naturals$ such that $T^k$ is the zero transformation. Prove that if $T$ is nilpotent, then $A$ is not invertible.
            \begin{solution}
                Assume $A$ is invertible. Let $k \in \naturals$ such that $T^k$ is the zero transformation. We know by part (a) that the standard matrix of $T^k$ is $A^k$. By problem 6c on Worksheet 6 (CHECK CITATION), the inverse of $A^k$ is $(A^{-1})^k$. However, the zero transformation has no inverse, so there is a contradiction. Thus $A$ cannot be invertible.
            \end{solution}
        \part Prove that if $T$ is nilpotent, then $A - I_n$ is invertible. [Hint: try multiplying out $(A - I_n)(-I_n - A - A^2 - \cdots - A^{k-1})$ and see what you get.]
            \begin{solution}
                Let $k \in \naturals$ such that $T^k$ is the zero transformation.\\
                Expanding $(A - I_n)(-I_n - A - A^2 - \cdots - A^{k-1})$:
                \begin{align*}
                    &= -A\left(I_n + A + \cdots + A^{k-1}\right) + I_n\left(I_n + A + \cdots + A^{k-1}\right) \tag{distributivity}\\
                    &= -\left(A + A^2 + \cdots + A^{k-1} + A^{k}\right) + \left(I_n + A + \cdots + A^{k-1}\right) \tag{distributivity} \\
                    &= \left(I_n + A + \cdots + A^{k-1}\right) - \left(A + A^2 + \cdots + A^{k-1} + 0_{n \times n}\right) \tag{$A^k = 0_{n \times n}$}\\
                    &= I_n
                \end{align*}
                Swapping the order of multiplication, 
                \begin{align*}
                    (-I_n &- A - A^2 - \cdots - A^{k-1})(A - I_n) \\
                    &= -\left(I_n + A + \cdots + A^{k-1}\right)A + \left(I_n + A + \cdots + A^{k-1}\right)I_n \tag{distributivity}\\
                    &= -\left(A + A^2 + \cdots + A^{k-1} + A^{k}\right) + \left(I_n + A + \cdots + A^{k-1}\right) \tag{distributivity} \\
                    &= \left(I_n + A + \cdots + A^{k-1}\right) - \left(A + A^2 + \cdots + A^{k-1} + 0_{n \times n}\right) \tag{$A^k = 0_{n \times n}$}\\
                    &= I_n
                \end{align*}
                So the inverse of $A - I_n$ is $-(I_n + A + A^2 + \cdots + A^{k-1})$, and $A - I_n$ is invertible by definition.
            \end{solution}
	\end{parts}

%2
\question Let $V$ be any vector space, and let $S$ be any set. Let $\mathcal{F}(S, V)$ denote the set of all functions from $S$ to $V$ . (Note: we are not assuming $S \subseteq V$ here, just that $S$ is some set. $S$ is not assumed to be a vector space, but it could be. Similarly, the functions in $F(S, V )$ are not assumed to be linear transformations, although it is possible that some of them might be.)
\par For any functions $f$, $g \in \mathcal{F}(S, V )$ we can define their sum to be the function $f + g$ given by the formula $(f + g)(s) = f (s) + g(s)$, where $s$ is any element in $S$. Similarly, for any scalar $c \in R$ and any function $f \in \mathcal{F}(S, V )$ we define the function $cf$ to be given by the formula $(cf )(s) = c(f (s))$ for all $s \in S$.
    \begin{parts}
        \part Prove that $\mathcal{F}(S, V )$ is a vector space. Note: For this problem you must explicitly prove that each of the vector space properties VS1-8 from Worksheet 6 is true. (These proofs should be very short but are not skippable.)
            \begin{solution}
                Let arbitrary $a, b \in \reals$, arbitrary $f, g, h \in \mathcal{F}(S, V )$. Note that $+_{\mathcal{F}(S, V )}$ borrows the qualities of $+_V$ (the summation operation of $V$) through the definition; namely associativity and commutativity. 
                \par VS-1: True, $(f + g) + h = (f(s) + g(s)) + h(s) =  f(s) + (g(s) + h(s)) = f + (g + h)$ by additive associativity of vector space $V$.\\
                VS-2: $f + g = f(s) + g(s) = g(s) + f(s) = g + f$ by addititve commutativity of vector space $V$. \\
                VS-3: True, $f(s) = 0_V$ satisfies this property. $g + f = g(s) + 0_V = g$ \\
                VS-4: True, for all values of $f(s)$, such a $-f(s)$ exists since $V$ is a vector space. So the function $-f$ exists as well. \\
                VS-5: True, $a(f+g) = a(f(s) + g(s)) = af(s) + ag(s) = af + ag$ by distributivity of the vector space $V$. \\
                VS-6: True, $(a + b)f = (a + b)f(s) = af(s) + bf(s) = af + bf$. by scalar multiplicative distributivity of vector space $V$. \\
                VS-7: True, $a(bf) = a(bf(s)) = (ab)f(s) = (ab)f$ by scalar multiplicative associativity of vector space $V$.
                VS-8: True, $1f = 1f(s) = f(s) = f$ by the unitary law of vector space V.
            \end{solution}
        \part Is $0_{\mathcal{F}(S,V )}$ the same element as $0_V$ ? If not, explain how they are different.
            \begin{solution}
                No. $0_{\mathcal{F}(S,V)}$ maps any element of the set $S$ to $0_V$, while $0_V$ is only a vector in the space $V$. $0_{\mathcal{F}(S,V)}$ is a function and can take an input, while $0_V$ cannot take an input like a function.
            \end{solution}
        \part We could similarly define $\mathcal{F}(V, S)$ to be the set of all functions from $V$ to $S$. Would $\mathcal{F}(V, S)$ also a vector space? Why or why not?
            \begin{solution}
                Not necessarily. We used the vector space properties of image $V$ to prove the vector space axioms for $\mathcal{F}(S, V)$. However, when arbitrary set $S$ is the image, those properties do not necessarily apply.
            \end{solution}
        \part The familiar vector spaces $\mathcal{P}$, $\mathcal{P}_n$ and $\mathcal{C}^\infty$ (all from Worksheet 6) are all subsets of $\mathcal{F}(S, V )$ for some $S$ and $V$ . What are $S$ and $V$ for each of these functions?
            \begin{solution}
                All of these vector spaces are composed of functions which map from $\reals \rightarrow \reals$.
            \end{solution}
    \end{parts}

%3
\question Let $\mathcal{P}$ be the vector space of all polynomial functions from $\reals$ to $\reals$ in the variable $t$, and for each $n \in \naturals$, let $\mathcal{P}_n$ be (as usual) the subset of $\mathcal{P}$ consisting of all polynomial functions of degree at most $n$. (We already know that $\mathcal{P}_n$ is also a vector space.) Also let $T : \mathcal{P} \rightarrow \mathcal{P}$ be the map defined by $T (p)(t) = p'(t) + p(0)$ for each $p \in \mathcal{P}$ and for all $t \in \reals$.
    \begin{parts}
        \part Show that $T$ is a linear transformation.
            \begin{solution}
                Note that the derivative is linear by problem 5 of worksheet 6. Although the current domain and codomain are $\mathcal{P}$ and not $\mathcal{C}^\infty$, the derivative is closed in $\mathcal{P} \subset \mathcal{C}^\infty$, so it is still linear.
                First, we show $T$ respects addition:
                \begin{align*}
                    T(p + q)(t) &= (p + q)'(t) + (p + q)(0) \tag{definition of $T$}\\
                    &= p'(t) + q'(t) + p(0) + q(0) \tag{linearity of derivative} \\
                    &= p'(t) + p(0) + q'(t) + q(0) \tag{associativity and commutativity of $\mathcal{P}$}\\
                    &= T(p) + T(q) \tag{definition of $T(p)$}
                \end{align*}
                Next, we show $T$ respects scalar multiplication. Let $c \in \reals$.
                \begin{align*}
                    T(cp) &= (cp)'(t) + (cp)(0) \tag{definition of $T$} \\
                    &= c(p'(t)) + cp(0) \tag{linearity of derivative} \\
                    &= c(p'(t) + p(0)) \tag{distributivity of scalar multiplication in vector space $\mathcal{P}$} \\
                    &= cT(p)
                \end{align*}
                Thus $T$ is linear.
            \end{solution}
        \part Let $n \in \naturals$, and let $T_n : \mathcal{P}_n \rightarrow \mathcal{P}_n$ be defined by $T_n(p)(t) = p'(t) + p(0)$, so that $T_n$ is just $T$ with both domain and codomain restricted to $\mathcal{P}_n$. Is $T_n$ injective? Is $T_n$ surjective?
        \part Is $T$ injective? Is $T$ surjective?
    \end{parts}
\end{questions}

\end{document}