\documentclass[12pt]{exam}
\usepackage{amsmath}
\usepackage{amssymb}
\usepackage{amsthm}
\usepackage{tikz}
\usepackage{mathtools}
\usepackage{graphicx}
\usepackage{wrapfig}

\usepackage{bm} %bold symbols
\usepackage{hyperref} %add links

%%%%%%%%%%%%%%%%%%%%%%%%%
% 	Define vars here 	%
%%%%%%%%%%%%%%%%%%%%%%%%%

\def\hwName{Homework Set Part B}
\author{Zhengyu James Pan} %use like \@author
\def\instructor{Dr. Paul Kessenich}
\def\email{jzpan@umich.edu}
\def\dueDate{Thurs, Feb 8}

\newcommand{\reals}{\mathbb{R}}
\newcommand{\naturals}{\mathbb{N}}
\newcommand{\ints}{\mathbb{Z}}
\newcommand{\transpose}{^\top}

\makeatletter

\begin{document}
%Header
\pagestyle{head}
\firstpageheader{}{}{}
\header{MATH 217}{\hwName}{\thepage}

%Solution formatting
\printanswers
\unframedsolutions

%Top matter
{\parindent0in
\bf
\begin{center}
	MATH 217 W24 - LINEAR ALGEBRA, Section 001 ({\instructor}) \\
	{\hwName} due {\dueDate} at 11:59pm \\
	\@author\ (\href{mailto:\email}{\email})
\end{center}
}

\begin{questions}
\question Let $\reals^n \xrightarrow{T} \reals^n$ be a linear transformation. As on HW 3, we define $T^k$ to be the $k$-fold composition of $T$ with itself. Let $A$ be the standard matrix of $T$, by which we mean the unique $n \times n$ matrix such that $T (\vec{x}) = A\vec{x}$ for all $\vec{x} \in \reals^n$.
	\begin{parts}
		\part Prove that for all $k$, the standard matrix for $T^k$ is the matrix $A^k$. [Hint: induction works nicely.]
		\begin{solution}
			We are given that the standard matrix $A^{(1)}$ represents the transformation $T^{(1)}$. Assume that the transformation $T^n$ can be represented by the standard matrix $A^n$. We know by a theorem on the worksheets that the standard matrix of two linear transformations, both from $\reals^n \rightarrow \reals^n$, is equal to the product of their respective standard matrices. Then $(T^n \circ T)(x) = A^n A \vec{x}$. This is equal to $T^{n+1}(x) = A^{n+1}\vec{x}$. So by induction, $T^k (\vec{x}) = A^k\vec{x}$ for all $\vec{x} \in \reals^n$ and $k \in \naturals$.
		\end{solution}
        \part We define $T$ to be nilpotent if there exists some $k \in \naturals$ such that $T^k$ is the zero transformation. Prove that if $T$ is nilpotent, then $A$ is not invertible.
            \begin{solution}
                
            \end{solution}
        \part Prove that if $T$ is nilpotent, then $A - I_n$ is invertible. [Hint: try multiplying out $(A - I_n)(I_n + A + A+2 + \cdots + A_{k-1})$ and see what you get.]
	\end{parts}
\end{questions}

\end{document}