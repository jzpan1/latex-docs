\documentclass[12pt]{exam}
\usepackage{amsmath}
\usepackage{amssymb}
\usepackage{amsthm}
\usepackage{tikz}
\usepackage{mathtools}
\usepackage{graphicx}
\usepackage{wrapfig}

\usepackage{bm} %bold symbols
\usepackage{hyperref} %add links

%%%%%%%%%%%%%%%%%%%%%%%%%
% 	Define vars here 	%
%%%%%%%%%%%%%%%%%%%%%%%%%

\def\hwName{Homework 3 Part B}
\author{Zhengyu James Pan} %use like \@author
\def\instructor{Dr. Paul Kessenich}
\def\email{jzpan@umich.edu}
\def\dueDate{Thursday, February 1}

\newcommand{\reals}{\mathbb{R}}
\newcommand{\naturals}{\mathbb{N}}
\newcommand{\ints}{\mathbb{Z}}
\newcommand{\transpose}{^\top}

\makeatletter

\begin{document}
%Header
\pagestyle{head}
\firstpageheader{}{}{}
\header{MATH 217}{\hwName}{\thepage}

%Solution formatting
\printanswers
\unframedsolutions

%Top matter
{\parindent0in
\bf
\begin{center}
	MATH 217 W24 - LINEAR ALGEBRA, Section 001 ({\instructor}) \\
	{\hwName} due {\dueDate} at 11:59pm \\
	\@author\ (\href{mailto:\email}{\email})
\end{center}
}

\begin{questions}
\question Determine whether the following statements are true or false, and justify your answer with a proof or a counterexample.
	\begin{parts}
		\part For all $2 \times 2$ matrices $A$ and $B$, $(AB)^T = A^T B^T$.
		\begin{solution}
            False. For example,  
            \begin{align*}
                A = \begin{bmatrix}
                    1 & 2 \\
                    3 & 4
                \end{bmatrix},&\ \ B = \begin{bmatrix}
                    5 & 6 \\
                    7 & 8
                \end{bmatrix} \\ \\
                (AB)\transpose &= \begin{bmatrix}
                    19 & 22 \\
                    43 & 50
                \end{bmatrix}\transpose \\
                &= \begin{bmatrix}
                    1 & 22 \\
                    43 & 50
                \end{bmatrix} \\
                \neq A\transpose B\transpose &= \begin{bmatrix}
                     1 & 3 \\
                     2 & 4
                \end{bmatrix}
                \begin{bmatrix}
                     5 & 7 \\
                     6 & 8
                \end{bmatrix} \\
                &= \begin{bmatrix}
                    23 & 31 \\
                    34 & 46
                \end{bmatrix}
            \end{align*}
		\end{solution}
        \part For all $2 \times 2$ matrices $A$ and $B$, $(AB)^\top \neq A\transpose B\transpose$.
            \begin{solution}
                False. For example:
                \begin{align*}
                    A = \begin{bmatrix}
                       1 & 3 \\
                       0 & 0
                    \end{bmatrix},&\ \ B = \begin{bmatrix}
                        1 & 1 \\
                        0 & 0
                     \end{bmatrix} \\ \\
                    (AB)\transpose &= \begin{bmatrix}
                        1 & 3 \\
                        0 & 0
                    \end{bmatrix}\transpose \\
                    &= \begin{bmatrix}
                        1 & 0 \\
                        3 & 0
                    \end{bmatrix} \\
                    = A\transpose B\transpose &= \begin{bmatrix}
                        1 & 0 \\
                        3 & 0
                    \end{bmatrix}
                    \begin{bmatrix}
                        1 & 0 \\
                        1 & 0
                    \end{bmatrix} \\
                    &= \begin{bmatrix}
                        1 & 0 \\
                        3 & 0
                    \end{bmatrix}
                \end{align*}
            \end{solution}
        \part For all matrices A and B such that the matrix product AB exists, $(AB)\transpose = B\transpose A\transpose$.
            \begin{solution}
                True. Computing with arbitrary matrices $A$ and $B$ with $ij$th element $a_{ij}, b{ij}$ respectively, we see the two products are identical.
                \begin{align*}
                    (AB)\transpose &= \begin{bmatrix}
                        a_{11} b_{11} + a_{12} b_{21} & a_{11} b_{12} + a_{12} b_{22} \\
                        a_{21} b_{11} + a_{22} b_{21} & a_{21} b_{12} + a_{22} b_{22}
                    \end{bmatrix}\transpose \\
                    &= \begin{bmatrix}
                        a_{11} b_{11} + a_{12} b_{21} & a_{21} b_{11} + a_{22} b_{21}\\
                        a_{11} b_{12} + a_{12} b_{22} & a_{21} b_{12} + a_{22} b_{22}
                    \end{bmatrix} \\\\
                    B\transpose A\transpose &= \begin{bmatrix}
                        b_{11} & b_{21} \\
                        b_{12} & b_{22}
                    \end{bmatrix}
                    \begin{bmatrix}
                        a_{11} & a_{21} \\
                        a_{12} & a_{22}
                    \end{bmatrix} \\
                    &= \begin{bmatrix}
                        b_{11} a_{11} + b_{21} a_{12} & b_{11} a_{21} + b_{21} a_{22}\\
                        b_{12} a_{11} + b_{22} a_{12} & b_{12} a_{21} + b_{22} a_{22}
                    \end{bmatrix}
                    &= \begin{bmatrix}
                        a_{11} b_{11} + a_{12} b_{21} & a_{21} b_{11} + a_{22} b_{21}\\
                        a_{11} b_{12} + a_{12} b_{22} & a_{21} b_{12} + a_{22} b_{22}
                    \end{bmatrix}
                \end{align*}
            \end{solution}
        \part If $A$ is a symmetric matrix, then for all $n \in \naturals$, $An$ is also symmetric.
            \begin{solution}
                True. If $A$ is symmetric, then $A = A\transpose$. Multiplying by $n$ on both sides results in $An = A\transpose n$. Since scalar multiplication applies to all elements of a matrix, transpose clearly respects scalar multiplication; thus $An = (An)\transpose$. This means $An$ is symmetric.
            \end{solution}
        \part If $A$ is a square matrix and $A^2$ is symmetric, then so is $A$.
            \begin{solution}
                False. The matrix $A = \begin{bmatrix}
                    -1 & -2 \\
                    2 & 1
                \end{bmatrix}$ is not symmetric. However, $A^2 = \begin{bmatrix}
                    -3 & 0 \\
                    0 & -3
                \end{bmatrix}$ is symmetric. This example demonstrates that $A$ is not necessarily symmetric if $A^2$ is.
            \end{solution}
    \end{parts}
%2
\question Determine whether the following statements are true or false, and justify your answer with a proof or a counterexample.
    \begin{parts}
        \part Every 3-by-3 matrix that has a row of zeros is not invertible.
            \begin{solution}
                A matrix with a row of zeros can only have 2 pivots, for a rank of 2. Thus by theorem 2.4.3, it is not invertible.
            \end{solution}
        \part Every square matrix with 1's down the main diagonal is invertible.
            \begin{solution}
                False. For example, $A= \begin{bmatrix}
                    1 & 2 \\
                    0.5 & 1
                \end{bmatrix}$ is not invertible: both $A\begin{bmatrix} 2 \\ 0 \end{bmatrix} = \begin{bmatrix} 2 \\ 1 \end{bmatrix} = A\begin{bmatrix} 0 \\ 1 \end{bmatrix}$
            \end{solution}
        \part For any matrix $A$, if $A$ is invertible, then so is $A^{-1}$.
            \begin{solution}
                True. PROOF NEEDED
            \end{solution}
        \part For any matrix $A$, if $A$ is invertible, then $A^n$ is invertible.
            \begin{solution}
                True. $\left(A^n\right)^{-1} = \left(A^{-1}\right)^n$ PROOF NEEDED
            \end{solution}
    \end{parts}


%3
\question Let $A$ be an $m \times n$ matrix. Prove that if there exists an $n \times m$ matrix $B$ such that $BA = I_n$, then the system of linear equations $A\vec{x} = \vec{0}$ has a unique solution. (Note: a matrix $B$ with this property is called a left-inverse for $A$. Can you guess why?)

%4 
\question Given two matrices A and B such that the product AB is defined (say, $A$ is $n \times m$ and $B$ is $m \times k$), exactly one of the following two statements is true:
    \begin{parts}
        \part Every column of $AB$ is a linear combination of columns of A,
        \part Every column of AB is a linear combination of columns of B.
    \end{parts}
    Prove the one that is true, and provide a counterexample for the one that is false.

\question Let $f : X \rightarrow X$ be a function. We let $f^n$ denote the function $f^n : X \rightarrow X$ given by
composing $f$ iteratively, $n$ many times. Also, we define $f^0$ to be the identity function, i.e. $\forall x \in X, f^0(x) = x$.
    \begin{parts}
        \part Assume that $X = \reals^d$. Prove by induction that if $f$ is a linear transformation, then the nth iterate $f^n$ is also a linear transformation.
        \part Find an example of a function $f : \reals^2 \rightarrow \reals^2$ which is not a linear transformation, but for
        which there exists an n such that the nth iterate $f^n$ is a linear transformation.
        \part Prove that for $X = Rd$ and $f$ linear, if the equation $f (x) = 0$ has a unique solution, then the iterated equation $f^n(x) = 0$ also has a unique solution.
    \end{parts}
\end{questions}

\end{document}