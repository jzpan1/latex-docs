\documentclass[12pt]{exam}
\usepackage{amsmath}
\usepackage{amssymb}
\usepackage{amsthm}
\usepackage{tikz}
\usepackage{mathtools}
\usepackage{graphicx}
\usepackage{wrapfig}

\usepackage{bm} %bold symbols
\usepackage{hyperref} %add links

%%%%%%%%%%%%%%%%%%%%%%%%%
% 	Define vars here 	%
%%%%%%%%%%%%%%%%%%%%%%%%%

\def\hwName{Homework Set Part B}
\author{Zhengyu James Pan} %use like \@author
\def\instructor{Dr. Paul Kessenich}
\def\email{jzpan@umich.edu}
\def\dueDate{???}

\newcommand{\reals}{\mathbb{R}}
\newcommand{\realsn}{\reals^{n}}
\newcommand{\nxn}{n\times n}
\newcommand{\realsnxn}{\reals^{\nxn}}
\newcommand{\naturals}{\mathbb{N}}
\newcommand{\ints}{\mathbb{Z}}
\newcommand{\transpose}{^\top}

\makeatletter

\begin{document}
%Header
\pagestyle{head}
\firstpageheader{}{}{}
\header{MATH 217}{\hwName}{\thepage}

%Solution formatting
\printanswers
\unframedsolutions

%Top matter
{\parindent0in
\bf
\begin{center}
	MATH 217 W24 - LINEAR ALGEBRA, Section 001 ({\instructor}) \\
	{\hwName} due {\dueDate} at 11:59pm \\
	\@author\ (\href{mailto:\email}{\email})
\end{center}
}

\begin{questions}
\question Question
	\begin{parts}
		\part Prove that $F$ is alternating if and only if $F (\vec u, \vec v) = -F (\vec v, \vec u)$ for all $\vec u, \vec v \in \reals^2$.
		\begin{solution}
			By bilinearity, we know 
            \begin{align*}
                F(u + v, v + u) &= 0 \\
                F(u, v + u) + F(v, v + u) &= 0 \\
                F(u, v) + F(u, u) + F(v, v) + F(v, u) &= 0 \\
                F(u, v) + 0 + 0 + F(v, u) &= 0 \\
                F(u, v) + F(v, u) &= 0 \\
                F(u, v) &= -F(v, u)
            \end{align*}
		\end{solution}
		\part Prove that if $F$ is alternating and $F (\vec e_1, \vec e_2) = 1$, then $F (\vec u, \vec v) = $ det$[\vec u \  \vec v]$ for all $\vec u, \vec v \in \reals^2$.
			\begin{solution}
				Express $\vec u$ and $\vec v$ as linear combinations of $e_1, e_2:$
				\[ \vec u = u_1 \vec e_1 + u_2 \vec e_2 \text{   and   } \vec v = v_1 \vec e_1 + v_2 \vec e_2 \]
				Then \begin{align*}
					F(\vec u, \vec v) &= F(u_1 \vec e_1 + u_2 \vec e_2, v_1 \vec e_1 + v_2 \vec e_2) \\
                    &= F(u_1 \vec e_1, v_1 \vec e_1 + v_2 \vec e_2) + F(u_2 \vec e_2, v_1 \vec e_1 + v_2 \vec e_2) \tag{bilinearity} \\
                    &= F(u_1 \vec e_1, v_1 \vec e_1) + F(u_1 \vec e_1, v_2 \vec e_2) + F(u_2 \vec e_2, v_1 \vec e_1 ) + F(u_2 \vec e_2, v_2 \vec e_2)\\
                    &= u_1 v_1 F( \vec e_1, \vec e_1) + u_1 v_2 F(\vec e_1, \vec e_2) + u_2 v_1 F(\vec e_2, \vec e_1 ) + u_2 v_2 F(\vec e_2, \vec e_2)\\
                    &= u_1 v_1 (0) + u_1 v_2 (1) + u_2 v_1 (-1) + u_2 v_2 (0) \tag{alternating} \\
                    &= u_1 v_2 - u_2 v_1\\
                    &= \text{det}\begin{bmatrix*}
                        u_1 & v_1 \\ u_2 & v_2
                    \end{bmatrix*} \\
                    &= \text{det}[\vec u\  \vec v]
				\end{align*}
			\end{solution}
	\end{parts}

\clearpage


\question \begin{parts}
    \part Prove that $T$ is a linear transformation.
        \begin{solution}
            Let $A, B \in \reals^{2 \times 2}$, and $c \in \reals$. \\
            $T$ respects addition:
            \[ T(A+B) = (A+B)M = AM + BM = T(A) + T(B) \]
            by distributivity of matrix multiplication.\\
            $T$ respects scalar multiplication:
            \[ T(cA) = (cA)M = c(AM) = cT(A) \]
            by properties of matrix multiplication.
            \par Since $T$ respects addition and scalar multiplication, it is linear.
        \end{solution}
    \part Find the $\mathcal E$-matrix $[T ]_\mathcal E$ of $T$, where $\mathcal E$ is the ordered basis
    \[ \mathcal E = \left(E_{11}, E_{12}, E_{21}, E_{22}\right) = \left( \begin{bmatrix} 1 & 0 \\ 0 & 0 \end{bmatrix}, \begin{bmatrix} 0 & 1 \\ 0 & 0 \end{bmatrix}, \begin{bmatrix} 0 & 0 \\ 1 & 0 \end{bmatrix}, \begin{bmatrix} 0 & 0 \\ 0 & 1 \end{bmatrix} \right) \]
    of $\reals^{2 \times 2}$. Your answer should be in terms of the entries of $M$.
        \begin{solution}
            \begin{align*}
                [T]_\mathcal E &= \begin{bmatrix} | & | & | & | \\ [T(E_{11})]_\mathcal E & [T(E_{12})]_\mathcal E & [T(E_{21})]_\mathcal E & [T(E_{22})]_\mathcal E \\ | & | & | & | \end{bmatrix}\\\\
                [T(E_{11})]_\mathcal E &= \left[\begin{bmatrix*} a & b \\ 0 & 0 \end{bmatrix*}\right]_\mathcal E = \begin{bmatrix} a \\ b \\ 0 \\ 0 \end{bmatrix} \\
                [T(E_{12})]_\mathcal E &= \left[\begin{bmatrix*} c & d \\ 0 & 0 \end{bmatrix*}\right]_\mathcal E = \begin{bmatrix} c \\ d \\ 0 \\ 0 \end{bmatrix} \\
                [T(E_{21})]_\mathcal E &= \left[\begin{bmatrix*} 0 & 0 \\ a & b \end{bmatrix*}\right]_\mathcal E = \begin{bmatrix} 0 \\ 0 \\ a \\ b \end{bmatrix} \\
                [T(E_{22})]_\mathcal E &= \left[\begin{bmatrix*} 0 & 0 \\ c & d \end{bmatrix*}\right]_\mathcal E = \begin{bmatrix} 0 \\ 0 \\ c \\ d \end{bmatrix} \\\\
                [T]_\mathcal E = \begin{bmatrix} a & c & 0 & 0 \\ b & d & 0 & 0 \\ 0 & 0 & a & c \\ 0 & 0 & b & d \end{bmatrix}
            \end{align*}
        \end{solution}
    \part 
\end{parts}
\end{questions}

\end{document}