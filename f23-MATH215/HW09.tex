\documentclass[12pt]{exam}
\usepackage{amsmath}
\usepackage{amssymb}
\usepackage{amsthm}
\usepackage{tikz}
\usepackage{mathtools}
\usepackage{graphicx}
\usepackage{wrapfig}

\usepackage{bm} %bold symbols
\usepackage{hyperref} %add links

%%%%%%%%%%%%%%%%%%%%%%%%%
% 	Define vars here 	%
%%%%%%%%%%%%%%%%%%%%%%%%%

\def\hwName{Homework Set 8: §16.1 – 16.3}
\author{Zhengyu James Pan} %use like \@author
\def\email{jzpan@umich.edu}
\makeatletter

\begin{document}
%Header
\pagestyle{head}
\firstpageheader{}{}{}
\header{MATH 215}{\hwName}{\thepage}

%Solution formatting
\printanswers
\unframedsolutions

%Top matter
{\parindent0in
\begin{center}
	\bf MATH 215 FALL 2023\\
	\bf \hwName \\
	\@author\ (\href{mailto:\email}{\email})
\end{center}
}

\begin{questions}
%1
\question Compute $\int_C x^2 y \, ds$, where $C$ is the segment of the helix of radius 1 about the $z$-axis, oriented counter-clockwise in the $xy$-plane, starting at (1, 0, 0) and ending at $(0, 1, \frac{\pi}{2})$.

%2
\question Compute $\int_C x^2 \, dx + y^2 \, dy$, where $C$ is the circular arc starting at (2, 0) and ending at (0, 2) followed by the straight line segment from (0, 2) to (-1, 1).

%3
\question Do Exercise 53 of §16.2 in $Stewart's\ Multivariable\ Calculus$.

%4
\question A wire has the shape of a helix with parametrization $x = t, y = 2 \cos t, z = 2 \sin t$ for $0 \leq t \leq 6\pi$, where distances are measured in cm. Find the mass and the center of mass of the wire if the density (in grams/cm) of the wire at any point is equal to four times the square of the distance from the origin to the point.

%5
\question Let $\overrightarrow{F} = \nabla f$ where $f(x, y) = \frac{y^{2002}}{1 + x^{200002} + y^{2002}}.$ Can you find a (smooth, simple, but not necessarily closed) curve C with the following property:
	\begin{parts}
		\part $\int_C \overrightarrow{F} \cdot d\overrightarrow{r} = \frac{1}{2}$
			\begin{solution}
				By the Fundamental Theorem of Line Integrals, this integral is equal to the value of $f(x_1, y_1)$ at the beginning of the curve subtracted from $f(x_2, y_2)$ at the end of the curve. So, we find two points where $f(x_2, y_2)  - f(x_1, y_1) = \frac{1}{2}$. (0, 0) and (0, 1) satisfy these conditions. So, the line segment from (0, 0) to (0, 1) fulfills this property.
			\end{solution}
		\part $\int_C \overrightarrow{F} \cdot d\overrightarrow{r} = 1$
		\begin{solution}
			The minimum value of $f$ is 0 at (0, 0), and only the limit at $(0, \infty)$ is 1. Thus, there is no finite curve where this is true. If an infinite curve is allowed, then the infinite line beginning at the origin along the $y$-axis satisfies this property. 
		\end{solution}
	\end{parts}

%6
\question Do Exercises 11, 31, and 32 of §16.3 in $Stewart's\ Multivariable\ Calculus$. \\
	\textbf{Solution}
	\begin{questions}
		\setcounter{question}{10}
		\question \begin{parts}
			\part The vector field is conservative, the gradient of the function $f(x, y) = x^2y$. Thus, the Fundamental Theorem of Line integrals applies. Since all these curves begin and end at the same point, the line integrals have the same value.
			\part $f(3, 2) - f(1, 2) = \boxed{16}$
		\end{parts}
		\setcounter{question}{30}
		\question 
		\question 
	\end{questions}

%7
\question 
	\begin{parts}
		\part Calculate $\oint_C \overrightarrow{F} \cdot d \overrightarrow{r}$ where
		\[ \overrightarrow{F}(x, y) = \langle \frac{2xy}{(x^2 + y^2)^2}, \frac{y^2 - x^2}{(x^2 + y^2)^2} \rangle \]
		for $(x, y) \neq (0, 0)$ and $C$ is the circle of radius $R$ centered at the origin, oriented clockwise.
			\begin{solution}
				We can parameterize this by $C(t) = \langle R\cos(t), R\sin(t) \rangle, 0 \leq t \leq \pi$. Then $C'(t) = \langle -R\sin(t), R\cos(t) \rangle$.
				The integral is then
				\begin{align*}
					\oint_C \overrightarrow{F} \cdot d \overrightarrow{r} &= \int_{0}^{2\pi} \langle \frac{2\sin(t)\cos(t)}{R^2}, \frac{\sin^2(t) - \cos^2(t)}{R^2} \rangle \cdot \langle -R\sin(t), R\cos(t) \rangle\, dt \\
					&= \frac{1}{R} \int_{0}^{2\pi} -2\sin^2(t)\cos(t) +\cos(t)\sin^2(t) - \cos^3(t)\, dt \\
					&= \frac{1}{R} \int_{0}^{2\pi} -\sin^2(t)\cos(t) - \cos^3(t)\, dt \\
					&= \frac{1}{R} \int_{0}^{2\pi} -\cos(t)(\sin^2(t) + \cos^2(t))\, dt \\
					&= \frac{1}{R} \int_{0}^{2\pi} -\cos(t)\, dt \\
					&= \boxed{0}
				\end{align*}
				We could also have noticed that $\overrightarrow{F}$ is the gradient of the function $f(x, y) = \frac{-y}{x^2 + y^2}$. Following from the Fundamental Theorem of Line Integrals, a line integral of a conservative vector field on a closed curve is 0. \qed
			\end{solution}
		\part Repeat the previous part, only this time take the curve C to be the ellipse defined by $4x^2 + 9y^2 = 36$, oriented counterclockwise. Hint: It may be possible to do this integration without parametrizing the ellipse.
		\begin{solution}
			$\overrightarrow{F}$ is the gradient of the function $f(x, y) = \frac{-y}{x^2 + y^2}$. Following from the Fundamental Theorem of Line Integrals, a line integral of a conservative vector field on a closed curve is \boxed{0}. \qed
		\end{solution}
	\end{parts}
\end{questions}

\end{document}