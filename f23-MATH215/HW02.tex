\documentclass[12pt]{exam}
\usepackage{amsmath}
\usepackage{amssymb}
\usepackage{amsthm}
\usepackage{tikz}
\usepackage{mathtools}
\usepackage{graphicx}
\usepackage{wrapfig}

\usepackage{hyperref} %add links

%%%%%%%%%%%%%%%%%%%%%%%%%
% 	Define vars here 	%
%%%%%%%%%%%%%%%%%%%%%%%%%

\def\hwName{Homework Set 2: §12.4 – 13.1}
\author{Zhengyu James Pan}
\def\email{jzpan@umich.edu}
\makeatletter

\begin{document}
%Header
\pagestyle{head}
\firstpageheader{}{}{}
\header{MATH 215}{\hwName}{\thepage}

%Solution formatting
\printanswers
\unframedsolutions

%Top matter
{\parindent0in
\begin{center}
	\bf MATH 215 FALL 2023\\
	\bf \hwName \\
	\@author\ (\href{mailto:\email}{\email})
\end{center}
}

\begin{questions}
%1
\question 
	\begin{parts}
		\part Find the equations of all planes parallel to the plane $y = 2$ 
			and 4 units away from it.
			\begin{solution}
				The plane $y = 2$ has a normal of $\langle 0,1,0\rangle$, and so will parallel planes. Adding 4 times this normal gives the plane $\boxed{y=6}$, and subtracting 4 times the normal gives $\boxed{y=-2}$.
				\qed
			\end{solution}
		\part Find the equations of all planes parallel to the plane 
			$2x - y + z = 0$ and 3 units away from it. Hint:
			Think about what it means for two planes to be parallel, 
			and how to find the distance between two planes.
			\begin{solution}
				Using the same approach as part b, we can find the normal direction and add or subtract it (scaled by the distance) from the coordinate vector in th vector equation. The normal of the given plane is $\mathbf{n} = \langle 2, -1, 1 \rangle$, and the unit vector is $\mathbf{e_n} = \langle \frac{2}{\sqrt{6}}, -\frac{1}{\sqrt{6}}, \frac{1}{\sqrt{6}} \rangle$. Adding 3 times this direction to $\langle x,y,z \rangle$ grants the equation \[\boxed{ 2x-y+z=-\frac{18}{\sqrt{6}} }.\] Subtracting grants \[\boxed{ 2x-y+z = \frac{18}{\sqrt{6}}} \tag*{\qed}\]
			\end{solution}
	\end{parts}
\clearpage
%2
\question Last week you found the surface equidistant between two points and 
	arrived at an equation for a plane. This week, find an equation for the 
	surface consisting of all points equidistant from the point (5, 0, 0) 
	and the plane y = 2. Identify what type of surface this is and sketch 
	it (do not use a plotting tool to generate the surface – I want to see 
	that you can sketch a surface from an equation. This might be a good
	skill to practice for a test, \textit{hint hint}.).
	\begin{solution}
		Similar to how the points equidistant from a line and a point in $\mathbb{R}^2$ is a parabola, points equidistant from a point and a plane will form a paraboloid in $\mathbb{R}^3$.
		\begin{align*}
			\{ \langle x,y,z \rangle : &y-2 = \pm \sqrt{(x - 5)^2 + y^2 + z^2} \} \\
			\{ \langle x,y,z \rangle : &y^2 - 4y + 4 = (x - 5)^2 + y^2 + z^2 \} \\
			&\boxed{\frac{(x - 5)^2}{4} + y + \frac{z^2}{4} = 1 } \tag*{\qed}
		\end{align*}
	\end{solution}
\clearpage
%3
\question Find parametric equations which describe the curve defined 
	by intersecting the cylinder $x^2 + y^2 = 64$ with the paraboloid of 
	revolution $x^2 + y^2 + 18z = 0$.
	\begin{solution}
		The cylinder can be parameterized as $(8\sin(t), 8\cos(t), z)$. Since the intersection must satisfy $x^2 + y^2 = 64$, then $64 + 18z = 0 \Rightarrow z= -\frac{9}{32}$. \\
		Thus the equation of the intersection is 
		\[\boxed{(8\sin(t), 8\cos(t), -\frac{9}{32})} \tag*{\qed}\]
	\end{solution}
\clearpage
%4
\question Find the equation of the line consisting of the points equidistant 
	from the three points $(2, 1, 1)$, $(-1, -1, 10)$, and $(1, 3, -4)$.
	\begin{solution}
		Since the equidistant points between pairs of points is a plane, we can find the intersection of two of the planes formed. \\
		Between the points $(2,1,1)$ and $(-1,-1,10)$ is the plane $\langle 3, 2, 9 \rangle \cdot (x - 0.5, y, z - 4.5) = 0$. \\
		Between the points $(-1, -1, 10)$ and $(1, 3, -4)$ is the plane $\langle 2, 4, -14 \rangle \cdot (x, y - 1, z - 3) = 0$. \\
		Crossing the two normals, we find the vector of the line is $ \langle -64, 60, 8 \rangle $. A single point on it would be the median of these three points, $(\frac{1}{3}, 1, \frac{7}{3})$. Thus, the vector equation of the line is:
		\[ \boxed{(x, y, z) = \left(\frac{1}{3}, 1, \frac{7}{3}\right) + \langle -16, 30, 4 \rangle} \tag*{\qed}\]
	\end{solution}
\clearpage
%5
\question 
	\begin{parts} 
		%a
		\part Describe and sketch the part of the first octant where 
		$x + 2y + 3z \leq 4$. (The first octant is the	region where $x, y, z \geq 0$).
			\begin{solution}
				This region will be the area under a plane, or alternatively a section of a rectangular prism "sliced" by the plane. This plane will have intercepts $(4, 0, 0)$, $(0, 2, 0)$, and $(0, 0, \frac{4}{3})$.
				\\\\\\\\\\\\\\\\\\\\\\\\\\ %sketch space
			\end{solution}
		\part Describe and sketch the region give\textbf{n} by 
		$|x| + 2|y| + 3|z| \leq 4$. It may help to look at the equation in each of the eight octants separately.
			\begin{solution}
				This will result in the figure from the previous problem in all 8 octants, rotated accordingly. This is because when crossing an octant, one of $x,y,z$ changes signs. However, due to the absolute value, the term that variable belonged to acts exactly like it did in the first octant, with all terms positive.
			\end{solution}
	\end{parts}
\clearpage
%6
\question Do Exercises 23-30 of section 12.6 of Stewart’s \textit{Multivariable Calculus}.
	\\\\\textbf{Solutions:}
	\begin{questions}
		\setcounter{question}{22}
		\question \boxed{VII} -- scaled by $\frac{1}{3}$ in $z$ direction and $\frac{1}{2}$ in $y$.
		\question \boxed{IV} -- same as problem 23 but switch $z$ and $x$.
		\question \boxed{II} -- can be rearranged to $y^2+1=x^2+z^2 \Rightarrow$ circles around y-axis which increase radius starting from 1 at the origin. Approaches a linear increase in radius, but not quite because of the $+1$ term. Symmetrical across $xz$ plane because $y$ is squared. Hyperboloid of one sheet
		\question \boxed{III} -- Same as 25, except the $-1$ causes no solutions until $y=1$. Thus hyperboloid of two sheets.
		\question \boxed{VI} -- A paraboloid shrunk by 2 in the $x$-direction.
		\question \boxed{I} -- A cone shrunk by 2 in the $x$-direction.
		\question \boxed{VIII} -- An ellipsoidal cylinder shrunk by 2 in the $x$-direction.
		\question \boxed{V} -- $x$ and $z$ have opposite signs, so hyperbolic paraboloid.
	\end{questions}
\clearpage
\setcounter{question}{6}
%7
\question
	\begin{parts}
		\part Sketch the curve defined by 
			$r(t) = \langle 5t \sin(3t), 6t \cos(3t), t^2\rangle$. 
			Find a quadratic surface on which this curve lives.
			\begin{solution}
				Let $t = \sqrt{z}$. Then all points on r(t) satisfy the equation
				$$\boxed{z^2 = \frac{x^2}{25} + \frac{y^2}{36}}$$
				\\\\\\\\\\\\\\\\\\\\\\\\\\ %sketch space
			\end{solution}
			
		\part Sketch the curve defined by 
			$r(t) = \langle 3 + 5 \sin(t^4), 2 - 6 \sin(t^4), 3 + 7 \sin(t^4) \rangle$. 
			Find a two-dimensional surface on which this curve lives 
			(there are an infinite number).
			\begin{solution}
				% TODO
				Since $\sin(t^4)$ is present in all three coordinates, we can reparameterize with $s = \sin(t^4)$, which simply gives us the parametric equation of a line, $\langle 3+5s, 2-6s, 3+7s \langle$. The plane then can be constructed using any orthogonal vector as the normal. Crossing $\langle 1,0,0 \rangle$ with $\langle 5,-6,7 \rangle$ grants $\langle 0, -7, -6 \rangle$, and using the point at $s = 0$ as a reference, we find the plane 
				\[ \boxed{\langle 0, -7, -6 \rangle \cdot (x - 3, y - 2, z - 3) = 0} \tag*{\qed} \]
			\end{solution}
	\end{parts}
\clearpage
%8
\question A cooling tower for a power plant is to be constructed in the shape
	 of a hyperboloid of one sheet, with equation given by 
	 $x^2/a^2 + y^2/b^2 - z^2/c^2 = 1$. The diameter of the circular base of the 
	tower is 80m, and the minimum diameter is is 40m, located 30m above the ground.
	\begin{parts}
		\part Sketch the tower. Make sure to label your sketch with relevant 
		diameters, heights, and the location of the origin.
		\begin{solution}
			\\\\\\\\\\\\\\\\\\\\\\\\\\ %sketch space
		\end{solution}

		\part Find an equation of the tower, i.e., find $a, b$, and $c$.
		\begin{solution}
			When $z = 0$, $\frac{x^2}{a^2} + \frac{y^2}{b^2} = 1$. We also know that the base is a circle, so $a$ and $b$ must be the same and be equal to the radius. Hence, $a, b = 40$. \\
			Additionally, at $z=30$, the radius is 20. Considering the point $(20, 0, 30)$, 
			\begin{align*}
				\frac{20^2}{40^2} - \frac{30^2}{c^2} &= 1 \\
				\frac{c^2}{4} - 30^2 &= c^2
				% TODO this isnt right
			\end{align*}
		\end{solution}
	\end{parts}
\clearpage

%9
\question Two jets travel simultaneously according to the vector equations
	$$\bold{r_1}(t) = \langle 2, -1, 3 \rangle + t \langle 4, 2, 8 \rangle 
	\text{ and }
	\bold{r_2}(t) = \langle -1, -3, 1\rangle + t \langle 1, 4, -2\rangle$$
	Let time $t$ be measured in seconds and positions measured in kilometers.
	\begin{parts}
		\part Find the shortest distance between the jets, and the time at 
			which this occurs.
		\part Find the shortest distance between the trajectories of the jets.
	\end{parts}
\clearpage
%10
\question Two people are located in space at the points $(0, -2, 1)$ and 
	$(1, 0, 2)$, respectively.
	\begin{parts}
		\part Find a parametric equation for the line of sight between the 
			two people.
		\part Between the two people is a hot air balloon in the shape of a 
			sphere, centered at $(0, -1, 1)$ and with radius 2. Can the two
			people see each other, or is the line of sight blocked by the 
			sphere? If the line of sight is blocked, find the points on the
			sphere that each person sees.
	\end{parts}

\end{questions}

\end{document}