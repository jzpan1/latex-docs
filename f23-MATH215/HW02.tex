\documentclass[12pt]{exam}
\usepackage{amsmath}
\usepackage{amssymb}
\usepackage{amsthm}
\usepackage{tikz}
\usepackage{mathtools}
\usepackage{graphicx}
\usepackage{wrapfig}

\usepackage{hyperref} %add links

%%%%%%%%%%%%%%%%%%%%%%%%%
% 	Define vars here 	%
%%%%%%%%%%%%%%%%%%%%%%%%%

\def\hwName{Homework Set 2: §12.4 – 13.1}
\author{Zhengyu James Pan}
\def\email{jzpan@umich.edu}
\makeatletter

\begin{document}
%Header
\pagestyle{head}
\firstpageheader{}{}{}
\header{MATH 215}{\hwName}{\thepage}

%Solution formatting
\printanswers
\unframedsolutions

%Top matter
{\parindent0in
\begin{center}
	\bf MATH 215 FALL 2023\\
	\bf \hwName \\
	\@author\ (\href{mailto:\email}{\email})
\end{center}
}

\begin{questions}
%1
\question 
	\begin{parts}
		\part Find the equations of all planes parallel to the plane $y = 2$ 
			and 4 units away from it.
			\begin{solution}
				The plane $y = 2$ has a normal of $\langle 0,1,0\rangle$, and so will parallel planes. Adding 4 times this normal gives the plane $\boxed{y=6}$, and subtracting 4 times the normal gives $\boxed{y=-2}$.
				\qed
			\end{solution}
		\part Find the equations of all planes parallel to the plane 
			$2x - y + z = 0$ and 3 units away from it. Hint:
			Think about what it means for two planes to be parallel, 
			and how to find the distance between two planes.
			\begin{solution}
				Using the same approach as part b, we can find the normal direction and add or subtract it (scaled by the distance) from the coordinate vector in th vector equation. The normal of the given plane is $\mathbf{n} = \langle 2, -1, 1 \rangle$, and the unit vector is $\mathbf{e_n} = \langle \frac{2}{\sqrt{6}}, -\frac{1}{\sqrt{6}}, \frac{1}{\sqrt{6}} \rangle$. Adding 3 times this direction to $\langle x,y,z \rangle$ grants the equation \[\boxed{ 2x-y+z=-\frac{18}{\sqrt{6}} }.\] Subtracting grants \[\boxed{ 2x-y+z = \frac{18}{\sqrt{6}}} \tag*{\qed}\]
			\end{solution}
	\end{parts}
\clearpage
%2
\question Last week you found the surface equidistant between two points and 
	arrived at an equation for a plane. This week, find an equation for the 
	surface consisting of all points equidistant from the point (5, 0, 0) 
	and the plane y = 2. Identify what type of surface this is and sketch 
	it (do not use a plotting tool to generate the surface – I want to see 
	that you can sketch a surface from an equation. This might be a good
	skill to practice for a test, \textit{hint hint}.).
\clearpage
%3
\question Find parametric equations which describe the curve defined 
	by intersecting the cylinder $x^2 + y^2 = 64$ with the paraboloid of 
	revolution $x^2 + y^2 + 18z = 0$.
\clearpage
%4
\question Find the equation of the line consisting of the points equidistant 
	from the three points $(2, 1, 1)$, $(-1, -1, 10)$, and $(1, 3, -4)$.
\clearpage
%5
\question 
	\begin{parts} 
		%a
		\part Describe and sketch the part of the first octant where 
		$x + 2y + 3z \leq 4$. (The first octant is the	region where $x, y, z \geq 0$).
		\part Describe and sketch the region give\textbf{n} by 
		$|x| + 2, |y| + 3, |z| \leq 4$. 
		It may help to look at the equation in each of the eight octants separately.
	\end{parts}
\clearpage
%6
\question Do Exercises 23-30 of section 12.6 of Stewart’s \textit{Multivariable Calculus}.
	\\\\\textbf{Solutions:}
	\begin{questions}
		\setcounter{question}{22}
		\question 
	\end{questions}
\clearpage
\setcounter{question}{6}
%7
\question
	\begin{parts}
		\part Sketch the curve defined by 
			$r(t) = \langle 5t \sin(3t), 6t \cos(3t), t^2\rangle$. 
			Find a quadratic surface on which this curve lives.

		\part Sketch the curve defined by 
			$r(t) = \langle 3 + 5 \sin(t^4), 2 - 6 \sin(t^4), 3 + 7 \sin(t^4) \rangle$. 
			Find a two-dimensional surface on which this curve lives 
			(there are an infinite number).
	\end{parts}
\clearpage
%8
\question A cooling tower for a power plant is to be constructed in the shape
	 of a hyperboloid of one sheet, with equation given by 
	 $x^2/a^2 + y^2/b^2 - z^2/c^2 = 1$. The diameter of the circular base of the 
	tower is 80m, and the minimum diameter is is 40m, located 30m above the ground.
	\begin{parts}
		\part Sketch the tower. Make sure to label your sketch with relevant 
		diameters, heights, and the location of the origin.

		\part Find an equation of the tower, i.e., find $a, b$, and $c$.
	\end{parts}
\clearpage

%9
\question Two jets travel simultaneously according to the vector equations
	$$\bold{r_1}(t) = \langle 2, -1, 3 \rangle + t \langle 4, 2, 8 \rangle 
	\text{ and }
	\bold{r_2}(t) = \langle -1, -3, 1\rangle + t \langle 1, 4, -2\rangle$$
	Let time $t$ be measured in seconds and positions measured in kilometers.
	\begin{parts}
		\part Find the shortest distance between the jets, and the time at 
			which this occurs.
		\part Find the shortest distance between the trajectories of the jets.
	\end{parts}
\clearpage
%10
\question Two people are located in space at the points $(0, -2, 1)$ and 
	$(1, 0, 2)$, respectively.
	\begin{parts}
		\part Find a parametric equation for the line of sight between the 
			two people.
		\part Between the two people is a hot air balloon in the shape of a 
			sphere, centered at $(0, -1, 1)$ and with radius 2. Can the two
			people see each other, or is the line of sight blocked by the 
			sphere? If the line of sight is blocked, find the points on the
			sphere that each person sees.
	\end{parts}

\end{questions}

\end{document}