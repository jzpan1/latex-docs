\documentclass[12pt]{exam}
\usepackage{amsmath}
\usepackage{amssymb}
\usepackage{amsthm}
\usepackage{tikz}
\usepackage{mathtools}
\usepackage{graphicx}
\usepackage{wrapfig}

\usepackage{bm} %bold symbols
\usepackage{hyperref} %add links

%%%%%%%%%%%%%%%%%%%%%%%%%
% 	Define vars here 	%
%%%%%%%%%%%%%%%%%%%%%%%%%

\def\hwName{Homework Set 5: §14.5 – 14.7}
\author{Zhengyu James Pan} %use like \@author
\def\email{jzpan@umich.edu}
\makeatletter

\begin{document}
%Header
\pagestyle{head}
\firstpageheader{}{}{}
\header{MATH 215}{\hwName}{\thepage}

%Solution formatting
\printanswers
\unframedsolutions

%Top matter
{\parindent0in
\begin{center}
	\bf MATH 215 FALL 2023\\
	\bf \hwName \\ 
	\@author\ (\href{mailto:\email}{\email})
\end{center}
}

\begin{questions}
%1
\question Below is a contour map for a function f:
\begin{center}
	\includegraphics*[scale=0.4]{images/05-contour.png}
\end{center}
	
	\begin{parts}
		\part  Suppose the function $f$ describes the topography of a park and you want to hike from around (-.7, .3) to around (.7, .2). Indicate a path that appears to minimize the cumulative elevation gain, defined as the sum of every gain in elevation throughout the hike.
			\begin{solution}
				\begin{center}
					\includegraphics*[scale=0.3]{images/05-contour-path.png}
				\end{center}
			\end{solution}
		\part Identify the locations of any maxima on the plot.
		\part Identify the locations of any minima on the plot.
		\part Identify the locations of any saddle points on the plot.
		\begin{solution}
			\begin{center}
				\includegraphics*[scale=0.5]{images/05-contour-labeledpoints.png}\\ 
				{\color{blue}Blue: minima;} {\color{red}Red: maxima;} {\color{black}Black: saddle points.}
			\end{center}
		\end{solution}
	\end{parts}
\clearpage
%2
\question One side of a triangle is increasing at a rate of 3 cm/s and a second side is decreasing at a rate of 2 cm/s. If the area of the triangle remains constant, at what rate does the angle between the sides change when the first side is 20 cm long, the second side is 30 cm, and the angle is $\pi/6$? $Hint$: This may not be a right triangle.
	\begin{solution}
		Let $a=20$ cm, $b = 30$ cm, and $\angle C=\frac{\pi}{6}$. We will use the $ab\sin(C)$ formula for the area of a triangle.
		\begin{gather*}
			5.483037238 = ab\sin(C) \\
			0 = (\frac{\delta a}{\delta t}b + \frac{\delta b}{\delta t}a)\sin(C)+ab\cos(C)\frac{\delta C}{\delta t}\\
			\frac{\delta C}{\delta t} = \frac{(\frac{\delta a}{\delta t}b + \frac{\delta b}{\delta t}a)\sin(C)}{ab\cos(C)} \\
			\boxed{\frac{\delta C}{\delta t} = 0.04811252243} \tag*{\qed}
		\end{gather*}
	\end{solution}
\clearpage
%3
\question If $x, y$, and $z$ are related by the equation $xz + y \ln(x - 1) + e^z = 1$, find $z, \delta z/\delta x,$ and $\delta z/\delta y$ when $(x, y) = (2, 1)$.
	\begin{solution}
		\begin{gather*}
			2z + 0 + e^z = 1\\
			\boxed{z=0} \\\\
			x\frac{\delta z}{\delta x} + z + \frac{y}{x - 1} + e^z\frac{\delta z}{\delta x} = 0 \\
			\frac{\delta z}{\delta x} = -\frac{\frac{y}{x - 1}}{x+e^z} \\
			\boxed{\frac{\delta z}{\delta x} = -\frac{1}{3}}\\\\
			xz\frac{\delta z}{\delta y} + \ln(x - 1) + e^z\frac{\delta z}{\delta y} = 0 \\
			\frac{\delta z}{\delta y} = \frac{\ln(x - 1)}{e^z + xz} \\
			\boxed{ \frac{\delta z}{\delta y} = 0} \tag*{\qed}
		\end{gather*}
	\end{solution}
\clearpage
%4
\question Newton's law of universal gravitation states that the (magnitude of the) gravitational force F between two objects is given by
\[ F = G\frac{m_1m_2}{r^2}, \]
where $G$ is the gravitational constant, $m_1$ and $m_2$ are the masses of the two objects, and $r$ is the distance between the two objects. Here $G = 6.674 \times 10^{- 11}$ m$^3$kg$^{-1}$s$^{-2}$. A team of amateur astronomers have
estimated that $m_1 = 7 \times 10^{24}$ kg, $m_2 = 3 \times 10^{23}$ kg, and $r = 2 \times 10^9$ m, with a maximum relative error of 0.2\%, 0.3\%, and 0.7\%, respectively, in the measurements. Use differentials to estimate the maximum
relative error (in percent) when the force F is calculated from these measurements.
	\begin{solution}
		\begin{gather*}
			dF = \frac{\delta F}{m_1}\Delta m_1 + \frac{\delta F}{m_2}\Delta m_2 + \frac{\delta F}{r}\Delta r\\
			dF = G\frac{m_2}{r^2}\Delta m_1 + G\frac{m_1}{r^2}\Delta m_2 + -2G\frac{m_1m_2}{r^3}\Delta r\\
			\boxed{\frac{dF(0.002, 0.003, 0.007)}{F} = 0.009 = 0.9\%} \tag*{\qed}
		\end{gather*}
	\end{solution}
\clearpage
%5
\question For the function $f(x, y) = x^2y^2 - y^3$, the point $P = (3, -1)$, and the point $Q = (-3, 5)$, find the
directional derivative of $f$ at $P$ in the direction of the vector $\overrightarrow{PQ}$ using
	\begin{parts}
		\part the definition of the directional derivative
			\begin{solution}
				
				\begin{gather*}
					e_{\overrightarrow{PQ}} = \langle -\frac{1}{\sqrt{2}}, \frac{1}{\sqrt{2}} \rangle \\
					{D_{\overrightarrow{PQ}}}f\left( {x,y} \right) = \mathop {\lim }\limits_{h \to 0} \frac{{f\left( x -\frac{1}{\sqrt{2}}h, y + \frac{1}{\sqrt{2}}h \right) - f\left( {x,y} \right)}}{h} \\
					D_{\overrightarrow{PQ}}f\left( x,y \right) = \mathop {\lim }\limits_{h \to 0} \frac{{\left( \left(x-\frac{1}{\sqrt{2}}h\right)^{2}\left(y+\frac{1}{\sqrt{2}}h\right)^{2}-\left(y+\frac{1}{\sqrt{2}}h\right)^{3} \right) - \left(x^2y^2 - y^3 \right)}}{h} \\
					D_{\overrightarrow{PQ}}f\left( x,y \right) = \mathop {\lim }\limits_{h \to 0} \frac{\left( \left(3-\frac{1}{\sqrt{2}}h\right)^{2}\left(-1+\frac{1}{\sqrt{2}}h\right)^{2}-\left(-1+\frac{1}{\sqrt{2}}h\right)^{3} \right) - 10}{h} \\
					D_{\overrightarrow{PQ}}f(x,y) = \mathop {\lim }\limits_{h \to 0} \frac{ \frac{h^4+44h^2-48\sqrt{2}h-8\sqrt{2}h^3+36 + 4 - 6\sqrt{2}h + 6h^2 - \sqrt{2}h^3}{4} - 10}{h} \\
					D_{\overrightarrow{PQ}}f(x,y) = \mathop {\lim }\limits_{h \to 0} \frac{h^3+44h-48\sqrt{2}-8\sqrt{2}h^2 - 6\sqrt{2} + 6h - \sqrt{2}h^2}{4} \\
					D_{\overrightarrow{PQ}}f(x,y) = \frac{-54\sqrt{2}}{4} = \boxed{ -\frac{27\sqrt{2}}{2} } \tag*{\qed}
				\end{gather*}
			\end{solution}
		\part a shortcut relying on the fact that f is differentiable
			\begin{solution}
				We dot the gradient with the direction.
				\begin{gather*}
					\nabla f = \langle 2xy^2, 2x^2y - 3y^2 \rangle \\
					e_{\overrightarrow{PQ}} \cdot \nabla f(3, -1) = -\frac{2xy^{2}}{\sqrt{2}}\ +\ \frac{\left(2x^{2}y-3y^{2}\right)}{\sqrt{2}} \\
					\boxed{D_{\overrightarrow{PQ}}=-\frac{27\sqrt{2}}{2}} \tag*{\qed}
				\end{gather*}
			\end{solution}
	\end{parts}
\clearpage
%6
\question Find and classify all critical points of the function $f (s, t) = 8s^4 + t^2 + st - 2s^2 - s^3$.
	\begin{solution}
		\begin{gather*}
			\nabla f = \langle 32s^3 + t - 4s - 3s^2, 2t + s \rangle \\
			\begin{cases}
				t = -\frac{s}{2}\\
				32s^3 + t - 4s - 3s^2 = 0
			\end{cases} \\
			32s^3 - \frac{9s}{2} - 3s^2 = 0 \\
			s(32s^2 - \frac{9}{2} - 3s) = 0 \\
			s = 0, \frac{3\pm \sqrt{9+576}}{64}
		\end{gather*}
		\begin{align*}
			(s, t) =    &\\
			(0, 0)&: \text{saddle}, \\
			\left(\frac{3 + \sqrt{585}}{64}, -\frac{3 + \sqrt{585}}{128}\right)&: \text{minimum,} \\ 
			\left(\frac{3 - \sqrt{585}}{64}, -\frac{3 - \sqrt{585}}{128}\right)&: \text{minimum}
			\tag*{\qed}
		\end{align*}
	\end{solution}
\clearpage
%7
\question Find the absolute maximum and minimum values of the function $f (x, y) = x^3 - 12y - y^3 + 3x$ on the triangle with vertices (2, 3), (2, -2), and (-2, 2).
	\begin{solution}
		\begin{center}
			\begin{tikzpicture}
				\draw [thin, gray, -stealth] (-4,0) -- (4,0);% x-axis
				\draw [thin, gray, -stealth] (0,-4) -- (0,4);% y-axis
				\draw[black, thick] (2,3) -- (2,-2);
				\draw[black, thick] (2,3) -- (-2,2);
				\draw[black, thick] (-2,2) -- (2,-2);
			\end{tikzpicture}
		\end{center}
		We first look for critical points within the triangle, then check the boundaries for maximums.
		\begin{gather*}
			\begin{cases}
				3x^2 + 3 = 0 \\
				3y^2 - 12 = 0 \\
			\end{cases} \\
			3x^2 = -3 \\
			x^2 = -1
		\end{gather*}
		Since $x$ would need to be imaginary, no general critical points exist. Thus, we look for extrema along the boundary lines $x=2$, $y=-x$, and $y=\frac{x}{4}$. 
		\begin{gather*}
			x=2: \\
			3y^2 - 12 = 0 \\
			y=\pm 2 \\
			f(2, 2) = -18 \\
			f(2, -2) = 46 \\
			\\
			y=-x: \\
			f(x) = 2x^3+15x \\
			\frac{df}{dx} = 6x^2+12+3=0 \\
			\text{No critical points, check bounding points:}\\
			f(-2) = -46 \\
			f(2) = 46 \\
			\\
			y=\frac{x}{4}: \\
			f(x) = \frac{65}{64}x^3=0 \\
			x=0 \\
			f(0) = 0 \\
			\text{check bounds:} \\
			f(2) = \frac{65}{8} \\
			f(-2) = -\frac{65}{8}
		\end{gather*}
		Thus, the absolute maximum is $\boxed{f(2, -2 = 46)}$ and the absolute minimum is
		\\ $\boxed{f(-2, 2) = 46}$. \qed
	\end{solution}
\clearpage
%8
\question The picture below shows the atmospheric pressure in hPa in Europe and the North Atlantic on October 11, 2022. You may find an atlas (or Google maps) helpful for this problem. Do not forget units.
\begin{center}
    \includegraphics*[scale=0.3]{images/05-map.png}
\end{center}
	\begin{parts}
		\part What is the atmospheric pressure in Dublin?
		\begin{solution}
			About 1024 hPa.
		\end{solution}
		\part Standing in Dublin, in what direction is the atmospheric pressure increasing the fastest? What is
		the rate of increase in this direction?
			\begin{solution}
				Atmospheric pressure is increasing the fastest in the direction slightly south of west, toward London (around the point labeled H, 1027). The distance to London is about 289 miles, or 465.1 km. Meanwhile, the distance to the 1020 contour is about 120 miles, or 193 km. Thus, we can estimate the rate of change as 
				\begin{align*}
					\frac{\frac{3}{465} + \frac{4}{193}}{2} = \boxed{0.01358850075\ \frac{\text{hPa}}{\text{km}}} \tag*{\qed}
				\end{align*}
			\end{solution}
		\part Walking due south from the center of Dublin at a speed of 2 m/s, at what rate is the atmospheric pressure changing along the route?
			\begin{solution}
				In the direction of due south, the rate of change is about 0.01242741612 hPa per km, estimated from the distance from the 1020 contour to the 1024 contour in the due south direction. Multiplying this by 0.120 km (equivalent to 2 m per second) grants \boxed{0.00149128993\text{ hPa per second.}} \qed
			\end{solution}
		\part What are the lowest and highest air pressures indicated on the map, and where are they located, roughly? Can you locate any saddle points?
			\begin{solution}
				The lowest pressure points are 982 hPa, near Norway; 989 hPa, near Greenland; and 992 hPa, on the east coast of Greenland. The highest pressure points are 1030 hPa, at the very bottom of the map; and 1027 hPa, near London.
			\end{solution}
	\end{parts}
\clearpage
\end{questions}

\end{document}