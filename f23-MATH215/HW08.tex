\documentclass[12pt]{exam}
\usepackage{amsmath}
\usepackage{amssymb}
\usepackage{amsthm}
\usepackage{tikz}
\usepackage{mathtools}
\usepackage{graphicx}
\usepackage{wrapfig}

\usepackage{bm} %bold symbols
\usepackage{hyperref} %add links

%%%%%%%%%%%%%%%%%%%%%%%%%
% 	Define vars here 	%
%%%%%%%%%%%%%%%%%%%%%%%%%

\def\hwName{Homework Set 8: §15.7 – 16.1}
\author{Zhengyu James Pan} %use like \@author
\def\email{jzpan@umich.edu}
\makeatletter

\begin{document}
%Header
\pagestyle{head}
\firstpageheader{}{}{}
\header{MATH 215}{\hwName}{\thepage}

%Solution formatting
\printanswers
\unframedsolutions

%Top matter
{\parindent0in
\begin{center}
	\bf MATH 215 FALL 2023\\
	\bf \hwName \\
	\@author\ (\href{mailto:\email}{\email})
\end{center}
}

\begin{questions}

%1
\question For the following problem, take $r, \theta, \rho$, and $\phi$ to have the standard definitions in cylindrical and spherical
coordinates. Describe (and try to sketch) the following surfaces:
	\begin{parts}
		\part $r = \theta$
			\begin{solution}
				A cylinder through a spiral starting from the origin.
				\begin{center}
					\includegraphics*[scale=0.5]{images/08-1a.png}
				\end{center}
			\end{solution}
		\part $\rho = \theta$
			\begin{solution}
				A spiral in the $xy$ plane, where each point has vertical arcs of circles passing through them to the line $x=y=0$, each with the origin as their center.
				\begin{center}
					\includegraphics*[scale=0.5]{images/08-1b.png}
				\end{center}
			\end{solution}
		\part $r = \rho$
			\begin{solution}
				The $xy$ plane.
				\begin{center}
					\includegraphics*[scale=0.5]{images/08-1c.png}
				\end{center}
			\end{solution}
		\part $\theta = \phi$
			\begin{solution}
				A curved surface. When a curve is drawn on this surface with $\rho$ fixed, the curve looks similar to a sin curve when viewed from the y-axis.
				\begin{center}
					\includegraphics*[scale=0.5]{images/08-1d.png}
				\end{center}
			\end{solution}
	\end{parts}

%2
\question Let $E$ be the ball of radius 1 centered at the point (0, 0, 1).
	\begin{parts}
		\part Show that $E$ is given in Cartesian coordinates by the equation $x^2 + y^2 + z^2 - 2z \leq 0$.
		\part Write $E$ in spherical coordinates. Make sure to specify the domain of $\rho, \theta,$ and $\phi$.
		\part Suppose the density on $E$ is proportional to the distance to the origin, with the largest density
		being equal to 2. Use spherical coordinates to compute the mass and center of mass of $E$.
		\part Suppose we tried to do this problem for the ball of radius 1 centered at the point (0, 1, 0). Why is this problem harder with the new ball?
	\end{parts}
%3
\question Begin with a sphere of radius $R$ and bore a hole into the sphere in the shape of a right circular cylinder,
leaving only a band of height $h$. Find the volume of the resulting shape.
	\begin{solution}
		The radius of the cylinder will be $r_c = \sqrt{R^2 - h^2}$. We use cylindrical coordinates to perform the integration.
		\begin{gather*}
			2\pi \int_{-h}^{h} \int_{\sqrt{R^2 - h^2}}^{\sqrt{R^2 - z^2}} r \, dr\, d\theta \\
			= \pi \int_{-h}^{h} \left(r^2\right)|_{\sqrt{R^2 - h^2}}^{\sqrt{R^2 - z^2}}  \, dr\, d\theta \\
			= \pi \int_{-h}^{h} R^2 - z^2 - R^2 + h^2 \, d\theta \\
			= \pi \left(-\frac{z^3}{3} + h^2 z \right)|_{z = -h}^{h} \\
			= \boxed{\frac{4\pi h^3}{3}} \tag*{\qed}
		\end{gather*}
	\end{solution}
%4
\question Find the mass of a wedge cut from a sphere of radius $R$ by two planes that intersect along a diameter and at an angle of $\frac{\pi}{5}$, assuming that the density is proportional to the distance from the origin in such a way that the maximum density is 2. (This shape should look like a segment of an orange.)
	\begin{solution}
		We use spherical coordinates for this problem, with $(r, \theta, \phi)$. The density function will be $\rho (r) = \frac{2 r}{R}$ to have a maximum density of 2 when the distance is equal to the radius.
		\begin{gather*}
			\frac{\pi}{5}\int_{0}^{R}\int_{0}^{\pi} \frac{2 r}{R} r^2 \sin(\phi)\, d\phi \, dr \\
			= \frac{\pi}{5R}\int_{0}^{R} 2 r^3 \int_{0}^{\pi} \sin(\phi)\, d\phi \, dr \\
			= \frac{\pi}{5R}\int_{0}^{R} 2 r^3 \left(-\cos(\phi)\right)|_{\phi = 0}^{\pi} \, dr \\
			= \frac{\pi}{5R}\int_{0}^{R} 4 r^3 \, dr \\
			= \frac{\pi}{5R} \left(r^4\right)|_{r=0}^{R} \\
			= \boxed{\frac{\pi R^3}{5}} \tag*{\qed}
		\end{gather*}
	\end{solution}

%5
\question Find $\int\int_R f(x, y)\, dA$ where $f(x, y) = 3y^2 - 4xy - 4x^2$ and $R$ is the quadrilateral with vertices (0, 2), (3, 0), (5, 4), and (2, 6). $Hint$: There may be a straightforward but tedious way to solve this problem, as well as a faster, more subtle, way to solve this problem.
	\begin{solution}
		We can factor $f(x, y) = (3y+2x)(y-2x)$. Then, we can use change of variables to change both the function and the bounds. Let $u = 3y + 2x$, $v = y - 2x$. Then $f(u, v) = uv$, $d(x, y) = \left(2 - 3(-2))\right)^{-1} d(u, v) = \frac{1}{8}d(u, v)$. Also, $R$ has vertices at $(u, v) = (6, 2), (6, -6), (22, -6), (22, 2)$.\\
		\begin{gather*}
			\frac{1}{8} \int_{-6}^{2} \int_{6}^{22} uv\sqrt{1 + u^2 + v^2} \, du\, dv \\
			t = 1 + u^2 + v^2, dt = 2u\, du \\
			= \frac{1}{16} \int_{-6}^{2} v \int_{u=6}^{u=22} \sqrt{t} \, dt\, dv \\
			= \frac{1}{16} \int_{-6}^{2} v  \left(\frac{2t^{3/2}}{3}\right)_{u=6}^{u=22}\, dv \\
			= \frac{1}{16} \int_{-6}^{2} v \left(\frac{2(485+v^2)^{3/2}}{3}\right) - v\left(\frac{2(37+v^2)^{3/2}}{3}\right) \, dv \\
			= \frac{1}{16} \int_{-6}^{2} v \left(\frac{2(485+v^2)^{3/2}}{3}\right) - v\left(\frac{2(37+v^2)^{3/2}}{3}\right) \, dv \\
			w = 485+v^2, dw = 2v\, dv; z = 37+v^2, dz = 2v\, dv \\
			= \frac{1}{16} \int_{v = -6}^{v = 2} \left(\frac{(w)^{3/2}}{3}\right) \, dw - \frac{1}{16} \int_{v = -6}^{v = 2} \left(\frac{(z)^{3/2}}{3}\right) \, dz \\
			= \frac{1}{8} \left(\frac{(w)^{5/2}}{15}\right)_{-6}^{v = 2} - \frac{1}{8} \left(\frac{(z)^{5/2}}{15}\right)_{v = -6}^{2} \\
			= \boxed{-7276.863857} \tag*{\qed}
		\end{gather*}
	\end{solution}
%6
\question Let $E$ be the region in the first quadrant that is above the line $y = \frac{x}{3}$, below the line $y = 3x$, and
between the curves defined by $xy = 3$ and $xy = 27$.
	\begin{parts}
		\part Sketch the region.
		\part Evaluate $\int \int (\frac{x^2}{y^2} + x^2 y^2) \, dA.$ (Hint: Try $u = xy$ and $v = \frac{y}{x}$.)
		\part Why was the hint a reasonable guess for a change of coordinates?
	\end{parts}
%7
\question Do Exercises 13-18 of §16.1 in \emph{Stewart's Multivariable Calculus}.
	\begin{solution}
		\begin{questions}
			\setcounter{question}{12}
			\question \boxed{IV} -- vectors with direction and magnitude equal to displacement, except flipped vertically.
			\question \boxed{V} -- downward direction when $x < y$, upward when $y < x$, horizontal when $x = y$.
			\question \boxed{I} -- when $y = -2$, vectors are horizontal.
			\question \boxed{VI} -- magnitude increases more with $x$ than $y$.
			\question \boxed{III} -- the magnitude/direction oscillates when either coordinate is fixed.
			\question \boxed{II} -- direction becomes more vertical when x increases, while horizontal component oscillates.
		\end{questions}
	\end{solution}
%8
\setcounter{question}{7}
\question Do Exercises 19-22 of §16.1 in \emph{Stewart's Multivariable Calculus}.
	\begin{solution}
		\begin{questions}
			\setcounter{question}{18}
			\question \boxed{IV} -- only constant vector field.
			\question \boxed{I} -- the vector field is constant when z is fixed.
			\question \boxed{III} -- always positive vertical direction, same direction as displacement from origin for $x$ and $y$.
			\question \boxed{II} -- same direction/magnitude as displacement from origin.
		\end{questions}
	\end{solution}
%9
\setcounter{question}{8}
\question Do Exercises 31-34 of §16.1 in \emph{Stewart's Multivariable Calculus}.
	\begin{solution}
		\begin{questions}
			\setcounter{question}{30}
			\question \boxed{III} -- gradient is $(2x, 2y)$, so linearly increasing magnitude and same direction as displacement from origin.
			\question \boxed{IV} -- gradient is $(2x + y, x)$, thus the direction is close to horizontal near the y-axis and becomes more vertical as x increases.
			\question \boxed{II} -- gradient is $(2x + 2y, 2y + 2x)$. Since the $x$ and $y$ coordinates are the same, the direction is always the same $\langle 1, 1 \rangle$, except with positive or negative magnitude.
			\question \boxed{I} -- Gradient will include something with $\cos$ for both $f_x$ and $f_y$ coordinates, thus the magnitude will oscillate.
		\end{questions}
	\end{solution}
\end{questions}

\end{document}