%April 26, 2011

% why did prof vig send me the lecture template i asked for the homework...
\documentclass[11pt]{article}
\usepackage[letterpaper,hmargin=0.5in,vmargin=.5in]{geometry}


\usepackage{amsmath,amsthm,amssymb,graphics,color}
\usepackage{hyperref} 
\usepackage{xcolor}
\usepackage[english]{babel}
\usepackage{tikz}
\usepackage[]{geometry}
\usetikzlibrary{quotes,angles}
\usepackage{graphicx}
%\usepackage[skip=2pt,font=scriptsize]{caption}
%\usepackage[colorlinks=true,
%inkcolor=blue]{hyperref}
%\usepackage{phaistos}


%\usepackage[latin9]{inputenc}



\usepackage{subcaption}
\usepackage{paralist}
\usepackage{graphicx}
\usepackage{esvect}
\usepackage{float}
\usepackage{array}
\usepackage{esvect}
\usepackage{float}
\usepackage{array}
\usepackage{tensor}
\usepackage{amsthm}
\usepackage{amsfonts}
\usepackage{bbm}
\usepackage{dsfont}



\newcommand*{\vertbar}{\rule[1ex]{0.5pt}{2.5ex}}
\newcommand*{\horzbar}{\rule[0.5ex]{2.5ex}{0.5pt}}
\newcommand{\C}{\mathbb{C}}
\newcommand{\Q}{\mathbb{Q}}
\newcommand{\Z}{\mathbb{Z}}
\newcommand{\N}{\mathbb{N}}
\newcommand{\D}{\mathbb{D}}
\newcommand{\R}{\mathbb{R}}
\newcommand{\F}{\mathbb{F}}
\newcommand{\T}{\mathbb{T}}
\newcommand{\ka}{\kappa}
\newcommand{\E}{\mathbb{E}}
\newcommand{\cn}{\text{cn}}
\newcommand{\sn}{\text{sn}}
\newcommand{\am}{\text{am}}
\newcommand{\res}{\text{res}}
\newcommand{\Id}{\text{Id}}
\newcommand{\im}{\text{Im}}
\newcommand{\Hom}{\text{Hom}}
\newcommand{\Ext}{\text{Ext}}
\newcommand{\Tor}{\text{Tor}}
\newcommand{\rk}{\text{rk}}
\newcommand{\Span}{\text{Span}}
\newcommand{\op}{\text{Op}}
\newcommand{\supp}{\text{supp}}
\newcommand{\dist}{\text{dist}}
\renewcommand{\phi}{\varphi}
\renewcommand{\epsilon}{\varepsilon}
\newcommand{\1}{\mathds{1}}
\newcommand{\var}{\text{Var}}
\newcommand{\cov}{\text{Cov}}

\newtheorem{theo}{Theorem}[section]
\newtheorem{prop}[theo]{Proposition}
\newtheorem{coro}[theo]{Corollary}
\newtheorem{lemm}[theo]{Lemma}
\newtheorem{conj}[theo]{Conjecture}
\newtheorem*{ques}{Question}
\newtheorem*{comm}{Comment}
\newtheorem*{resp}{Response}
\theoremstyle{definition}
\newtheorem{def1}[theo]{Definition}
\theoremstyle{remark}
\newtheorem{rema}[theo]{Remark}



\newcommand{\Op}{\operatorname{Op}}
\newcommand{\U}{\mathcal{U}}
\newcommand{\nwc}{\newcommand}
\nwc{\Oph}{\operatorname{Op}_\hbar}
\nwc{\la}{\langle}
\nwc{\ra}{\rangle}

\nwc{\mf}{\mathbf} %Latex (as in \bf not tilted math letters)
\nwc{\blds}{\boldsymbol} %Latex 
\nwc{\ml}{\mathcal} %Latex


\renewcommand{\Im}{\operatorname{Im}}
\renewcommand{\Re}{\operatorname{Re}}


\newcommand{\rr}{r}
\newcommand{\inv}{^{-1}}
\newcommand{\szego}{Szeg\"o\ }
\newcommand{\sm}{\setminus}
\newcommand{\kahler}{K\"ahler\ }
\newcommand{\sqrtn}{\sqrt{N}}
\newcommand{\wt}{\widetilde}
\newcommand{\wh}{\widehat}
\newcommand{\dd}{\\ \\ \\ \noindent}
\newcommand{\dds}{\\ \\ \\ \\ \\ \\ \noindent}
\newcommand{\al}{\noindent}
\newcommand{\ex}{\al \textbf{Example:} }

\renewcommand{\d}{\partial}
\newcommand{\dbar}{\bar\partial}
\newcommand{\ddbar}{\partial\dbar}
\newcommand{\sech}{\mbox{sech}}


\newcommand{\half}{{\frac{1}{2}}}
\newcommand{\vol}{{\operatorname{Vol}}}
\newcommand{\codim}{{\operatorname{codim\,}}}
\newcommand{\SU}{{\operatorname{SU}}}
\newcommand{\FS}{{{\operatorname{FS}}}}
\newcommand{\corr}{{\operatorname{Corr}}}


\renewcommand{\phi}{\varphi}
\newcommand{\eqd}{\buildrel {\operatorname{def}}\over =}
\newcommand{\nhat}{\raisebox{2pt}{$\wh{\ }$}}


\newcommand{\go}{\mathfrak}



\newtheorem{maintheo}{{\sc Theorem}}
\newtheorem{mainprop}{{\sc Proposition}}
\newtheorem{mainlem}{{\sc Lemma}}
\newtheorem{maincor}{{\sc Corollary}}
\newtheorem{maindefin}{{\sc Definition}}
%\newtheorem{theo}{{\sc Theorem}}
\newtheorem{cor}[theo]{{\sc Corollary}}
\newtheorem{lem}[theo]{{\sc Lemma}}
%\newtheorem{prop}[theo]{{\sc Proposition}}
\newenvironment{example}{\medskip\noindent{\it Example:\/} }{\medskip}
\newenvironment{rem}{\medskip\noindent{\it Remark:\/} }{\medskip}
\newenvironment{defn}{\medskip\noindent{\it Definition:\/} }{\medskip}
\newenvironment{claim}{\medskip\noindent{\it Claim:\/} }{\medskip}
\addtolength{\baselineskip}{1pt}



\newcommand{\pink}[1]{{\color{pink}{#1}}}
\newcommand{\blue}[1]{{\color{blue}{#1}}}
\newcommand{\red}[1]{{\color{red}{#1}}}
\newcommand{\brown}[1]{{\color{brown}{#1}}}
\newcommand{\black}[1]{\color{black}}

\newcommand{\vs}{\vspace{0.2in}}
\linespread{1.25}


\newcommand{\np}{\newpage \al}

%\title{Math 425 Notes}
%\date{}


%\author{}
%\address{Department of Mathematics, University of Michigan, Ann Arbor, MI 48109, USA} \email{vig@umich.edu}



\begin{document}
	%\maketitle
	\pagenumbering{gobble}
	
	\al \large \textbf{{Week 4, Lectures 9,10,11}}
	\\
	\\
	\textbf{Game plan for this week:} Functions of multiple variables!
	\\
	\\
	\textbf{Review:}
	\\
	\\
	\textbf{Velocity, speed and acceleration:}
	% \begin{figure}[h!]
	% 	\includegraphics[scale = 0.6]{accel1}
	% 	\includegraphics[scale = 0.6]{accel2}
	% \end{figure}\\
	\textbf{Centripetal Force:}
	% \begin{figure}[h!]
	% 	\includegraphics[scale = 0.7]{cptl}
	% \end{figure}\\
	\textbf{Ballistic motion:}
	% \begin{figure}[h!]
	% 	\includegraphics[scale = 0.7]{ballistic}
	% \end{figure}\\
	\textbf{Normal and tangential acceleration:}\\
	% \begin{figure}[h!]
	% 	\includegraphics[scale = 0.7]{nta}
	% \end{figure}
	\newpage
	\al \textbf{Functions:}\\
	% \begin{figure}[h!]
	% 	% \includegraphics[scale = 0.75]{functions}
	% \end{figure}\\
	\textbf{Domain and range:}
	\dds
	\textbf{Independent and dependent variables:}
	\dds
	\ex My dialing in recipe for espresso:\\
	\dd
	\ex What are the domain and range of $f(x) = \sqrt{9 - x^2 - y^2}$?\\
	\np
	\textbf{Graphs:}\\
	% 	\begin{figure}[h!]
	% 	\includegraphics[scale = 0.75]{graph}
	% \end{figure}\\
	\textbf{Level sets:} also called countour lines\\
	% \begin{figure}[h!]
	% 	\includegraphics[scale = 0.55]{ct3}
	% 	\includegraphics[scale = 0.5]{level}
	% \end{figure}\\
	\textbf{Functions of $3$ variables:}
	\dds
	\textbf{Notation:} In general, it is cumbersome to write $(x,y,z)$ or $(x_1, x_2, \cdots, x_n)$.
	\np
	\textbf{A collage of beautiful contour lines:}
	% \begin{figure}[h!]
	% 	\includegraphics[scale = 0.25]{ct1}
	% 	\includegraphics[scale = 0.25]{ct2}\\
	% 	\includegraphics[scale = 0.45]{ct4}
	% 		\includegraphics[scale = 0.25]{ct6}\\
	% 		\includegraphics[scale = 0.45]{ct5}
	% \end{figure}
	\np
	\textbf{Limits:} (briefly)
	% 	\begin{figure}[h!]
	% 	\includegraphics[scale = 0.75]{continuity}
	% \end{figure}\\
	\textbf{Continuity:}
	\dd
	\textbf{Partial derivatives:}
	% 	\begin{figure}[h!]
	% 	\includegraphics[scale = 0.45]{partials}\\
	% 	\includegraphics[scale = 0.75]{partialgraph}
	% \end{figure}\\
\np
	\ex What are the partial derivatives of $\sin x \cos y$?
	\dds
	\ex If we have an implicitly defined function for $z = f(x,y)$ according to the equation $x^3 + y^3 + z^3 + 6xyz = 1$, find the partials of $f$.
	\dds
	\textbf{Higher order partials:}
	\dds
	\textbf{Clairaut's theorem:}
	\dds
	\textbf{A glance at PDE:}
	\np
	\textbf{Nodal sets:} (for fun)
	% \begin{figure}[h!]
	% 	\includegraphics[scale = 0.25]{nodal1}
	% 	\includegraphics[scale = 0.25]{nodal2}
	% 			\includegraphics[scale = 0.5]{nodal3}
	% 	\includegraphics[scale = 0.4]{nodal4}
	% 			\includegraphics[scale = 0.4]{nodal5}
	% 			\includegraphics[scale = 0.4]{nodal6}
	% 			\includegraphics[scale = 1.5]{steve}
	% \end{figure}
\np
	\textbf{Tangent planes:}
	% 		\begin{figure}[h!]
	% 	\includegraphics[scale = 0.75]{tangent}\\
	% 	\includegraphics[scale = 0.45]{tangent2}
	% \end{figure}\\
	\textbf{Equation of the tangent plane:}
	% \begin{figure}[h!]
	% 	\includegraphics[scale = 0.45]{tangenteq}
	% \end{figure}\\
	\textbf{Linearization:} (really affine)
	\\ \dd
	\textbf{Continuity:} If derivatives exist and are continuous near a point, then the function is differentiable there.
	\np
	\textbf{Total differential:}\dds\\\\
	% \includegraphics[scale = 0.5]{dz}
	\\
	\\
	\ex If $f(x,y) = 3x^2 + 2xy + y^4$, find the differential $df$.
	
	%	\includegraphics[scale = 0.7]{lln}
\end{document}
