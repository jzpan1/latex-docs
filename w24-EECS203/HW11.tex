\documentclass[12pt]{exam}

% essential packages
\usepackage{fullpage} % margin formatting
\usepackage{enumitem} % configure enumerate and itemize
\usepackage{amsmath, amsfonts, amssymb, mathtools} % math symbols
\usepackage{xcolor, colortbl} % colors, including in tables
\usepackage{makecell} % thicker \Xhline in table
\usepackage{graphicx} % images, resizing

% sometimes needed packages
\usepackage{hyperref} % hyperlinks
% \hypersetup{colorlinks=true, urlcolor=blue}
% \usepackage{logicproof} % natural deduction
\usepackage{tikz} % drawing graphs
\usetikzlibrary{positioning}
\usepackage{multicol}
\usepackage{algpseudocode} % pseudocode

% paragraph formatting
\setlength{\parskip}{6pt}
\setlength{\parindent}{0cm}

% newline after Solution:
\renewcommand{\solutiontitle}{\noindent\textbf{Solution:}\par\noindent}

% less space before itemize/enumerate
\setlist{topsep=0pt}

% creates \filcl to grey out cells for groupwork grading
\newcommand{\filcl}{\cellcolor{gray!25}}

% creates \probnum to get the problem number
\newcounter{probnumcount}
\setcounter{probnumcount}{1}
\newcommand{\probnum}{\arabic{probnumcount}. \addtocounter{probnumcount}{1}}

% use roman numerals by default
\setlist[enumerate]{label={(\roman*)}}

% creates custom list environments for grading guidelines, question parts
\newlist{guidelines}{itemize}{1}
\setlist[guidelines]{label={}, left=0pt .. \parindent, nosep}
\newlist{gwguidelines}{enumerate}{1}
\setlist[gwguidelines]{label={(\roman*)}, nosep}
\newlist{qparts}{enumerate}{2}
\setlist[qparts]{label={(\alph*)}}
\newlist{qsubparts}{enumerate}{2}
\setlist[qsubparts]{label={(\roman*)}}
\newlist{stmts}{enumerate}{1}
\setlist[stmts]{label={(\roman*)}, nosep}
\newlist{pflist}{itemize}{4}
\setlist[pflist]{label={$\bullet$}, nosep}
\newlist{enumpflist}{enumerate}{4}
\setlist[enumpflist]{label={(\arabic*)}, nosep}

\printanswers

\newcommand{\prevhwnum}{10}
\newcommand{\hwnum}{11}

\begin{document}
%%%%%%%%%%%%%%% TITLE PAGE %%%%%%%%%%%%%%%
\title{EECS 203: Discrete Mathematics\\
  Winter 2024\\
  Homework \hwnum{}}
\date{}
\author{}
\maketitle
\vspace{-50pt}
\begin{center}
  \huge Due \textbf{Tuesday, April 23}, 10:00 pm\\
\Large No late homework accepted past midnight.\\
\vspace{10pt}
\large Number of Problems: $6+0$
\hspace{3cm}
Total Points: $100+0$
\end{center}
\vspace{25pt}
\begin{itemize}
    \item \textbf{Match your pages!} Your submission time is when you upload the file, so the time you take to match pages doesn't count against you.
    \item Submit this assignment (and any regrade requests later) on Gradescope. 
    \item Justify your answers and show your work (unless a question says otherwise).
    \item By submitting this homework, you agree that you are in compliance with the Engineering Honor Code and the Course Policies for 203, and that you are submitting your own work.
    \item Check the syllabus for full details.
\end{itemize}
\newpage
%%%%%%%%%%%%%%% TITLE PAGE %%%%%%%%%%%%%%% 

\section*{Individual Portion}

\subsection*{\probnum Big-$O$reo [15 points]}
Give the tightest big-O estimate for each of the following functions. Justify your answers.
\begin{qparts}
    \item $f(n) = (2^n + n^n)\cdot (n^3 + n \log n^n)$
    \item $g(n) = (n^n + n!)\cdot (n + 1)\hspace{0.1in} + \hspace{0.1in}(n^3 + 3^n)\cdot(\sqrt{n} + \log n)$
    \item $h(n) = (n^n + n^2)\cdot(n^n + n)\hspace{0.1in} + \hspace{0.1in}(\log 3 + n^n)\cdot(n^2 + n^n)$
\end{qparts}
\begin{solution}
    \begin{qparts}
        \item $n^n$ and $n^3$ dominate respectively. Thus \[f(n) = O(n^n n^3) = O(n^{n+3})\]
        \item In the left product, $n^n$ and $n$ dominate. In the right product, $3^n$ and $\sqrt n$ dominate. But the left product will dominate over the right product. So \[g(n) = O(n^n n) = O(n^{n+1})\]
        \item In both products, $n^n$ and $n^n$ dominate. So $2n^{2n}$ will dominate the sum. So \[h(n) = O(n^{2n})\].
    \end{qparts}
\end{solution}

\subsection*{\probnum On the Run [20 points]}
Give the tightest big-O estimate for the number of operations (where an operation is arithmetic, a comparison, or an assignment) used in each of the following algorithms. \textbf{Explain your reasoning.}

\begin{qparts}

    \item \begin{algorithmic}
        \Function{doubleTrouble}{$a_1,\dots,a_N\in\mathbb{R}, j\in\mathbb{R}$}
            \State{$j \gets 1$}
            \For{$i\coloneqq 1\text{ to }N$}
                \If{$i = j$}
                    \State{$j \gets 2j$}
                \EndIf
            \EndFor
            \State\Return{$j$}
        \EndFunction
    \end{algorithmic}

    \item \begin{algorithmic}
        \Function{sumSquares}{$N \in \mathbb{Z}^+$}
            \If{$N = 1$}
                \State\Return{$1$}
            \EndIf
            \State{$value \gets \textsc{sumSquares}(N - 1) + N^2$}
            \State\Return{$value$}
        \EndFunction
    \end{algorithmic}

    \item \begin{algorithmic}
        \Function{findLTMinProduct}{$a_1,\dots,a_N\in\mathbb{R}$}
            \State{$p\gets 203$}
            \For{$i\coloneqq 1\text{ to }N$}
                \For{$j\coloneqq 1\text{ to }N$}
                    \If{$a_ia_j < p$}
                        \State{$p\gets a_ia_j$}
                    \EndIf
                \EndFor
            \EndFor 
            \State{$numLTMinProduct\gets 0$}
            \For{$k\coloneqq 1\text{ to }N$}
                \If{$a_k < p$}
                    \State{$numLTMinProduct \gets numLTMinProduct + 1$}
                \EndIf
            \EndFor
        \State\Return{$numLTMinProduct$}
        \EndFunction
    \end{algorithmic}

    \item \begin{algorithmic}
        \Function{subtractAndAdd}{$N\in\mathbb{Z}$}
            \While{$N > 0$}
                \If{$N$ is even}
                    \State{$N \gets N - 3$}
                \EndIf
                \If{$N$ is odd}
                    \State{$N \gets N + 1$}
                \EndIf
            \EndWhile
            \State\Return{$N$}
        \EndFunction
    \end{algorithmic}

    \item \begin{algorithmic}
        \Function{search}{$a_1,\dots,a_N \in \mathbb{R}, target \in \mathbb{R}$}
            \State{$left\gets 1$}
            \State{$right\gets N$}
            \While{$\text{True}$}
                \State{$mid\gets \lfloor \frac{left + right}{2} \rfloor$}
                \If{$a_{mid} = target$}
                    \State\Return{$mid$}
                \EndIf
                \If{$right \leq left$}
                    \State\Return{$-1$}
                \EndIf
                \If{$a_{mid} < target$}
                    \State{$left\gets mid + 1$}
                \EndIf
                \If{$a_{mid} > target$}
                    \State{$right\gets mid - 1$}
                \EndIf
            \EndWhile
        \EndFunction
    \end{algorithmic}
\end{qparts}

\begin{solution}
    \begin{qparts}
        \item The loop executes $N$ times, and contains no loops within itself. So this function is $O(N)$.
        \item The function calls itself recursively with $N-1$ until it reaches $1$. So it is $O(N)$.
        \item The first loop executes $N$ times, containing another loop which executes $N$ times, for a total of $N^2$. The last loop executes $N$ times. So the $N^2$ dominates, and the function is $O(N^2)$.
        \item This loop will take at most $\dfrac{N+1}2$ steps, if $N$ is odd. (The first loop will add one. Then, every subsequent loop will do -3, +1 = -2.) If $N$ is even, it will take at most $\frac{N}2$ steps. (Every loop will do -3, +1 = -2.) So this function will be $O(N)$.
        \item The loop ends when $a_{mid} = target$ or $right \leq left$. Every loop, the midpoint is set to the floor of the midpoint between the left and right indexes. If the target is not found, the left or right index moves next to the midpoint, and the loop runs again. So every loop, the search area is halved. Thus the loop will take at most $\log_2 N + C$ steps. So the time complexity will be $O(\log N)$.
    \end{qparts}
\end{solution}


\subsection*{\probnum This one's bound to be fun! [18 points]}
You are given the following bounds on functions $f$ and $g$:
\begin{itemize}
    \item $f(x)$ is $O(203^xx^2)$ and $\Omega(3^x\log x)$
    \item $g(x)$ is $O(\frac{x!}{2^x})$ and $\Omega(4^x)$
\end{itemize}
Find the following, simplify your answer as much as possible.
\begin{qparts}
    \item Find the tightest big-O and big-$\Omega$ estimates that can be \textit{guaranteed} of $f(x)(g(x))^2$. 
    \item Find the tightest big-O and big-$\Omega$ estimates that can be \textit{guaranteed} of $f(x)+g(x)$.
    \item Let $h(x)=f(x)-g(x)$. Prove or disprove that $h(x)$ is $\Omega(4^x)$.
\end{qparts}

\begin{solution}
    \begin{qparts}
        \item For big-O, we find the parts which will dominate the function. This will be 
        \[ f(x)(g(x))^2 = O(203^xx^2) \cdot O\left(\frac{x!}{2^x}\right)^2 = O\left( \left(\frac{203}{4}\right)^x (x!)^2x^2\right)\]
        For $\Omega$, we compute using the $\Omega$ portions of the respective functions. 
        \[ f(x)(g(x))^2 = \Omega(3^x\log x) \cdot \Omega(4^{2x}) = \Omega(48^x \log x) \]
        \item Since this is addition, we simply select the dominating parts between each added function. This will be $O\left(\dfrac{x!}{2^x}\right)$ and $\Omega(4^x)$.
        \item By the rule of scalar multiplication, the scalar multiplication by $-1$ does not change the $\Omega$ of $g(x)$. Then, by the addition rule, we take the dominating $\Omega$ of $f$ and $g$, which is $4^x$. So by big-$\Omega$ properties, $h(x) = \Omega(4^x)$.
    \end{qparts}
\end{solution}

\subsection*{\probnum Big Function Fun [16 points]}
Prove or disprove the following: 
\begin{qparts}
    \item If $f(x)$ is $O(g(x))$ then $2^{f (x)}$ is $O(2^{g(x)}).$
    \item If $f(x)$ is $O(g(x))$ then $(f(x))^2$ is $O\big((g(x))^2\big).$
\end{qparts}
Note that in these proofs you do not need to use the definition of big-O, but can use the properties for combining mathematical functions covered in lecture.

\begin{solution}
    \begin{qparts}
        \item False. For instance, take $f(x) = 3x$, $g(x) = x$. Then $f(x) = O(g(x))$, but $2^{f (x)} = 8^x$ is not $O(2^{g(x)}) = O(2^x)$.
        \item By the product property for big-O, $(f_1 \cdot f_2)(x) =  O(g_1(x) \cdot g_2(x))$ for $f_1(x) = O(g_1(x))$ and $f_2(x) = O(g_2(x))$. So the statement is true if we take $f_1(x) = f_2(x) = f(x)$, $g_1(x) = g_2(x) = g(x)$ in the product property.
    \end{qparts}
\end{solution}


\subsection*{\probnum Roots and Shoots [16 points]}
Suppose $f$ satisfies $f(n) = 2f(\sqrt{n}) + \log_2 n$, whenever $n$ is a perfect square greater than 1, and additionally satisfies $f(2) = 1$.
\begin{qparts}
    \item Find $f(16)$.
    \item Find a big-O estimate for $g(m)$ where $g(m) = f(2^m).$
    
    \textbf{Hint:} Make the substitution $m = \log_2 n.$
    \item Find a big-O estimate for $f(n)$. 
\end{qparts}

\begin{solution}
    \begin{qparts}
        \item \begin{align*}
            f(16) &= 2f(4) + \log_2 16 \\
            &= 4f(2) + 2\log_2 4 + 4 \\
            &= 4 + 4 + 4 \\
            &= 12
        \end{align*}
        \item Substituting $\log_2 n = m$, 
            \begin{align*}
                g(m) &= f(2^m) \\
                &= 2f(2^{m/2}) + m \\
                &= 2g\left(\frac m2 \right) + m \\
            \end{align*}
            By the Master Theorem, this recursive relation is $O(m \log m)$.
        \item Using the previous part and substituting $m = \log_2 n$, we find 
            \[ f(n) = O\left(\log_2 (n) \log (\log_2 n) \right) \]
    \end{qparts}
\end{solution}


\subsection*{\probnum GG Brown Laboratory [15 points]}
What is the tightest big-O bound on the runtime complexity of the following algorithm?
\begin{algorithmic}
\Function{badsearch}{$n$}
    \If{$n\geq 1$}
        \State{$\textsc{badsearch}(\lfloor \frac n3\rfloor)$}
        \For{$i\coloneqq 1\text{ to }n$}
            \For{$j\coloneqq 1\text{ to }\lfloor \frac n2\rfloor$}
                \State{\textbf{print} ``Hello I am lost"}
            \EndFor
        \EndFor
        \State{$\textsc{badsearch}(\lfloor \frac n3\rfloor)$}
        \State{\textbf{print} ``Nevermind I got it"}
    \EndIf
\EndFunction
\end{algorithmic}

\begin{solution}
    This function does not run any commands if $n$ is greater than 1. If it is greater than 1, then it will run itself with $n/3$ twice, and also run the middle loops, which run a total of $n^2 / 2$ commands at most. So we have the recursive relation $B(n)$ for the number of commands which the function executes:
        \[B(n) = 2B\left(\frac{n}3 \right) + \frac{n^2}2 \]
    Then by the Master Theorem, this function is $\boxed{O(n^2)}$, since $2 < 3^2$.
\end{solution}



\end{document}
