\documentclass[12pt]{exam}

% essential packages
\usepackage{fullpage} % margin formatting
\usepackage{enumitem} % configure enumerate and itemize
\usepackage{amsmath, amsfonts, amssymb, mathtools} % math symbols
\usepackage{xcolor, colortbl} % colors, including in tables
\usepackage{makecell} % thicker \Xhline in table
\usepackage{graphicx} % images, resizing

% sometimes needed packages
\usepackage{hyperref} % hyperlinks
% \hypersetup{colorlinks=true, urlcolor=blue}
% \usepackage{logicproof} % natural deduction
% \usepackage{tikz} % drawing graphs
% \usetikzlibrary{positioning}
% \usepackage{multicol}
% \usepackage{algpseudocode} % pseudocode

% paragraph formatting
\setlength{\parskip}{6pt}
\setlength{\parindent}{0cm}

% newline after Solution:
\renewcommand{\solutiontitle}{\noindent\textbf{Solution:}\par\noindent}

% less space before itemize/enumerate
\setlist{topsep=0pt}

% creates \filcl to grey out cells for groupwork grading
\newcommand{\filcl}{\cellcolor{gray!25}}

% creates \probnum to get the problem number
\newcounter{probnumcount}
\setcounter{probnumcount}{1}
\newcommand{\probnum}{\arabic{probnumcount}. \addtocounter{probnumcount}{1}}

% use roman numerals by default
\setlist[enumerate]{label={(\roman*)}}

% creates custom list environments for grading guidelines, question parts
\newlist{guidelines}{itemize}{1}
\setlist[guidelines]{label={}, left=0pt .. \parindent, nosep}
\newlist{gwguidelines}{enumerate}{1}
\setlist[gwguidelines]{label={(\roman*)}, nosep}
\newlist{qparts}{enumerate}{2}
\setlist[qparts]{label={(\alph*)}}
\newlist{qsubparts}{enumerate}{2}
\setlist[qsubparts]{label={(\roman*)}}
\newlist{stmts}{enumerate}{1}
\setlist[stmts]{label={(\roman*)}, nosep}
\newlist{pflist}{itemize}{4}
\setlist[pflist]{label={$\bullet$}, nosep}
\newlist{enumpflist}{enumerate}{4}
\setlist[enumpflist]{label={(\arabic*)}, nosep}

\printanswers

\newcommand{\reals}{\mathbb{R}}
\newcommand{\naturals}{\mathbb{N}}
\newcommand{\ints}{\mathbb{Z}}
\newcommand{\transpose}{^\top}

\newcommand{\prevhwnum}{2}
\newcommand{\hwnum}{3}

\begin{document}
%%%%%%%%%%%%%%% TITLE PAGE %%%%%%%%%%%%%%%
\title{EECS 203: Discrete Mathematics\\
  Winter 2024\\
  Homework \hwnum{}}
\date{}
\author{}
\maketitle
\vspace{-50pt}
\begin{center}
  \huge Due \textbf{Thursday, Feb. 8}, 10:00 pm\\
\Large No late homework accepted past midnight.\\
\vspace{10pt}
\large Number of Problems: $7+1$
\hspace{3cm}
Total Points: $100+30$
\end{center}
\vspace{25pt}
\begin{itemize}
    \item \textbf{Match your pages!} Your submission time is when you upload the file, so the time you take to match pages doesn't count against you.
    \item Submit this assignment (and any regrade requests later) on Gradescope. 
    \item Justify your answers and show your work (unless a question says otherwise).
    \item By submitting this homework, you agree that you are in compliance with the Engineering Honor Code and the Course Policies for 203, and that you are submitting your own work.
    \item Check the syllabus for full details.
\end{itemize}
\newpage
%%%%%%%%%%%%%%% TITLE PAGE %%%%%%%%%%%%%%% 

\subsection*{\probnum On the Contrary [12 points]}
Let $n$ be an integer. Prove that if $4\,|\,(n^2-1),$ then $n$ is odd using
\begin{qparts}
    \item a proof by contraposition, and
    \item a proof by contradiction.
\end{qparts}
Then,
\begin{qparts}[resume]
    \item compare your answers to parts (a) and (b). What is different? What is the same?
\end{qparts}


\begin{solution}
    \begin{qparts}
        \item We will prove the contrapositive. Assume $n$ is even. Then $n$ can be expressed as $2m$, where $m$ is an integer. Then $n^2-1 = (2m)^2 - 1 = 4m^2 - 1$. $m^2$ is an integer, so $n^2-1$ is not divisible by 4. Thus the contrapositive is true, and the original statement is also true.
        \item Assume $4\,|\,(n^2-1)$ and $n$ is even. Then $n$ can be expressed as $2m$, where $m$ is an integer. Then $n^2-1 = (2m)^2 - 1 = 4m^2 - 1$. $m^2$ is an integer, so $n^2-1$ is not divisible by 4. This contradicts our original assumption, so the assumption is false.
        \item The assumption for contradiction contains one more clause. The contrapositive proof is to prove a truth, while the contradiction proves falsity. However, most of the steps taken to prove these two are the same, such as using $n = 2m$.
    \end{qparts}
\end{solution}


\subsection*{\probnum An Even-Numbered Question about Even Numbers [16 points]} 

\textbf{Prove or disprove} the following statements:

\begin{qparts}
    \item For all integers $x$, if $x$ is even, then $x^2$ is even.
    \item For all integers $x$, if $x^2$ is even, then $x$ is even.
    \item For all integers $x$, if $x$ is even, then $2x$ is even.
    \item For all integers $x$, if $2x$ is even, then $x$ is even.
\end{qparts}

\begin{solution}
    \begin{qparts}
        \item Assume $x$ is even. Then $x$ can be expressed $2m$, where $m$ is an integer. Then $x^2 = (2m)^2 = 4m^2 = 2(2m^2)$. $2m^2$ is an integer since $m$ is an integer, so $x^2$ is even.  
        \item We will prove the contrapositive. Assume $x$ is odd. Then $x$ can be written $2k + 1$ where $k$ is an integer. Then $x^2$ can be expressed $(2k+1)^2 = 4k^2+ 4k + 1 = 2(2k^2 + 2k) + 1$. So $x^2$ is not even when $x$ is odd, and the given statement is true.
        \item $2x$ is always even by definition, so the implication is true.
        \item False. Consider $x=1$. $2x =2$ is even, but $x$ is odd.
    \end{qparts}
\end{solution}


\subsection*{\probnum Even Stevens [16 points]}
\textbf{Prove or disprove} the following statement: ``There is a finite amount of even numbers."
\begin{solution}
    Assume the statement is true, and let $2n$ be the largest of these even numbers, where $n$ is an integer. Then $2n + 2 = 2(n+1)$ is even by definition, so $2n$ was not the largest even number and there is a contradiction. Thus the statement is false.
\end{solution}


\subsection*{\probnum Pay it Forward (Or Don't, It's Up To You) [12 points]}
Consider a centipede game, where there are two players: Ka-chun and Zyaire. The game starts by Ka-chun's decision of take or wait.

\begin{itemize}
    \item If Ka-chun takes, Ka-chun earns \$1 while Zyaire earns nothing, and the game ends.
    \item If Ka-chun waits, then Zyaire can choose between take or wait. If Zyaire takes, Zyaire earns \$2 while Ka-chun earns nothing and the game ends. If Zyaire waits it becomes Ka-chun's turn to choose again.
    \item If they keep waiting the reward grows by \$1 each round, until Zyaire's choice of taking \$20 or waiting, when the game will end no matter what.
\end{itemize} Both of Ka-chun and Zyaire want to maximize their rewards, and behave as perfect logicians.

\begin{qparts}
    \item Suppose Ka-chun and Zyaire made it to round 20. What happens in round 20? 
    \item Using your answer to (a), what would happen if they made it to round 19?
    \item Building off of parts (a) and (b), argue that Ka-chun should take \$1 in the very first round.
\end{qparts}

\begin{solution}
    \begin{qparts}
        \item In round 20, Zyaire is required to take the \$20.
        \item In round 19, Ka-chun knows Zyaire will take the \$20 next round, so he will take \$19 in this round to maximize his reward, instead of getting nothing next round.
        \item In round 18, Zyaire will have the same reasoning as Ka-chun in round 19, so he would take the money in this round. Ka-chun, knowing this, would then take the money in round 17. This pattern extends similarly until the first round, where Ka-chun knows Zyaire will to take the money in the next round. So, Ka-chun should take the \$1 in the first round.
    \end{qparts}
\end{solution}


\subsection*{\probnum Proofs to the Max [12 points]}

Prove that for all real numbers $a$, $b$, and $c$, if $\max\left \{a^2(b-c), -a\right \}$ is non-negative, then $a\leq 0$ or $b\geq c$.

\textbf{Note:} You can use the following facts in your proof:
\begin{itemize}
    \item If $x$ and $y$ are positive, then $x\cdot y$ is positive.
    \item If $x$ is positive and $y$ is negative, then $x\cdot y$ is negative.
    \item If $x$ and $y$ are negative, then $x\cdot y$ is positive.
\end{itemize}

\begin{solution}
    Assume $\max\left \{a^2(b-c), -a\right \}$ is non-negative. We will use casework. There are 2 possible cases, $\max\left \{a^2(b-c), -a\right \} = a^2(b-c)$ or $\max\left \{a^2(b-c), -a\right \} = -a$.
    \begin{itemize}
        \item \par (Case 1) Assume $\max\left \{a^2(b-c), -a\right \} = a^2(b-c)$. Then $a^2(b-c)$ is nonnegative.
        \begin{itemize}
            \item If $a^2(b-c) = 0$, one or both of the factors is 0 by the zero product property (0 can only be the product of 0 and some other number). If $a^2 = 0$, then $a = 0 \leq 0$ and the statement is true. If $b-c = 0$, then $b=c$ and the statement is true.
            \item If $a^2(b-c) > 0$, both $a^2$ and $b-c$ are simultaneously positive or negative by the given facts. It is impossible for $a^2$ to be negative by the given facts; $a \cdot a$ is either positive$\cdot$positive or negative$\cdot$negative. This would result in a contradiction if both factors were negative, so they must both be positive. 
            \par If they are positive, then $b-c > 0 \Rightarrow b \geq c$, which satisfies the original statement. 
        \end{itemize}
        \item (Case 2) Assume $\max\left \{a^2(b-c), -a\right \} = -a$. Then $-a \geq 0$. Adding $a$ to both sides, we find $0 \geq a$, making the statement true.
    \end{itemize}
    We have shown that the statement holds for all cases, so it is true.
\end{solution}


\subsection*{\probnum Let's All Be Rational [16 points]}

Show that these statements about a real number $x$ are equivalent to each other:
\begin{stmts}
    \item $x$ is rational
    \item $\frac{x}{2}$ is rational
    \item $3x-1$ is rational.
\end{stmts}

\textbf{Hint:} One way to prove statements (i), (ii) and (iii) are equivalent is by proving (i) $\rightarrow$ (ii), (ii) $\rightarrow$ (iii), and (iii) $\rightarrow$ (i).

\begin{solution}
    Assume $x$ is rational. Then it can be expressed $\frac{p}{q}$, where $p, q \in \ints$ are relatively prime and $q$ is nonzero. Then $\frac{x}{2} = \frac{p}{2q}$. Since $2q \neq 0 \in \ints$, this is also a rational number. So (i) $\rightarrow$ (ii).
    \par Assume $\frac{x}{2}$ is rational. Then it can be expressed $\frac{p}{q}$, where $p, q \in \ints$ are relatively prime and $q$ is nonzero. Then $3x-1 = \frac{6p}{q} - 1 = \frac{6p - q}{q}$. Since $p$ and $q$ are integers, $6p - q$ is an integer. Also, $q \neq 0$. Thus $3x - 1$ is rational by definition. So (ii) $\rightarrow$ (iii).
    \par Assume $3x - 1$ is rational. Then it can be expressed $\frac{p}{q}$, where $p, q \in \ints$ are relatively prime and $q$ is nonzero. Then $x = \frac{(3x - 1)+1}{3} = \frac{\frac{p+q}{q}}{3} = \frac{p+q}{3q}$. Since $p$ and $q$ are integers, $p+q$ and $3q$ are integers, $3q \neq 0$, and $x$ is rational by definition. So (iii) $\rightarrow$ (i).
    \par We have proven (i) $\rightarrow$ (ii), (ii) $\rightarrow$ (iii), and (iii) $\rightarrow$ (i). Thus (i), (ii) and (iii) are equivalent.
\end{solution}


\subsection*{\probnum Irrational Pr00f [16 points]}
Prove or disprove that the product of a nonzero rational number and an irrational number is irrational.

\begin{solution}
    Let $r = \frac{a}{b}$ be a nonzero rational number with $a,b \in \ints$, with $a, b \neq 0$. Let $t$ be an irrational number. Assume the product $rt$ is rational and can be expressed $rt = \frac{p}{q}$. Then \begin{align*}
        t = \frac{rt}{r} &= \frac{\frac{p}{q}}{\frac{a}{b}} \\
        &= \frac{bp}{aq}
    \end{align*}
    $bp, aq \in \ints$ because $b, p, a, q \in \ints$. This means $t$ is rational, which is a contradiction. Thus, our assumption that $rt$ is rational was impossible. So the statement is true.
\end{solution}

\end{document}
