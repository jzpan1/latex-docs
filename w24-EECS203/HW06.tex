\documentclass[12pt]{exam}

% essential packages
\usepackage{fullpage} % margin formatting
\usepackage{enumitem} % configure enumerate and itemize
\usepackage{amsmath, amsfonts, amssymb, mathtools} % math symbols
\usepackage{xcolor, colortbl} % colors, including in tables
\usepackage{makecell} % thicker \Xhline in table
\usepackage{graphicx} % images, resizing

% sometimes needed packages
\usepackage{hyperref} % hyperlinks
% \hypersetup{colorlinks=true, urlcolor=blue}
% \usepackage{logicproof} % natural deduction
% \usepackage{tikz} % drawing graphs
% \usetikzlibrary{positioning}
% \usepackage{multicol}
% \usepackage{algpseudocode} % pseudocode

% paragraph formatting
\setlength{\parskip}{6pt}
\setlength{\parindent}{0cm}

\newcommand{\reals}{\mathbb{R}}
\newcommand{\naturals}{\mathbb{N}}
\newcommand{\ints}{\mathbb{Z}}
\newcommand{\transpose}{^\top}

% newline after Solution:
\renewcommand{\solutiontitle}{\noindent\textbf{Solution:}\par\noindent}

% less space before itemize/enumerate
\setlist{topsep=0pt}

% creates \filcl to grey out cells for groupwork grading
\newcommand{\filcl}{\cellcolor{gray!25}}

% creates \probnum to get the problem number
\newcounter{probnumcount}
\setcounter{probnumcount}{1}
\newcommand{\probnum}{\arabic{probnumcount}. \addtocounter{probnumcount}{1}}

% use roman numerals by default
\setlist[enumerate]{label={(\roman*)}}

% creates custom list environments for grading guidelines, question parts
\newlist{guidelines}{itemize}{1}
\setlist[guidelines]{label={}, left=0pt .. \parindent, nosep}
\newlist{gwguidelines}{enumerate}{1}
\setlist[gwguidelines]{label={(\roman*)}, nosep}
\newlist{qparts}{enumerate}{2}
\setlist[qparts]{label={(\alph*)}}
\newlist{qsubparts}{enumerate}{2}
\setlist[qsubparts]{label={(\roman*)}}
\newlist{stmts}{enumerate}{1}
\setlist[stmts]{label={(\roman*)}, nosep}
\newlist{pflist}{itemize}{4}
\setlist[pflist]{label={$\bullet$}, nosep}
\newlist{enumpflist}{enumerate}{4}
\setlist[enumpflist]{label={(\arabic*)}, nosep}

\printanswers

\newcommand{\prevhwnum}{5}
\newcommand{\hwnum}{6}

\begin{document}
%%%%%%%%%%%%%%% TITLE PAGE %%%%%%%%%%%%%%%
\title{EECS 203: Discrete Mathematics\\
  Winter 2024\\
  Homework \hwnum{}}
\date{}
\author{}
\maketitle
\vspace{-50pt}
\begin{center}
  \huge Due \textbf{Thursday, Mar. 14th}, 10:00 pm\\
\Large No late homework accepted past midnight.\\
\vspace{10pt}
\large Number of Problems: $7+2$
\hspace{3cm}
Total Points: $100+30$
\end{center}
\vspace{25pt}
\begin{itemize}
    \item \textbf{Match your pages!} Your submission time is when you upload the file, so the time you take to match pages doesn't count against you.
    \item Submit this assignment (and any regrade requests later) on Gradescope. 
    \item Justify your answers and show your work (unless a question says otherwise).
    \item By submitting this homework, you agree that you are in compliance with the Engineering Honor Code and the Course Policies for 203, and that you are submitting your own work.
    \item Check the syllabus for full details.
\end{itemize}
\newpage
%%%%%%%%%%%%%%% TITLE PAGE %%%%%%%%%%%%%%% 

\section*{Individual Portion}

\subsection*{\probnum Mod Warm-up [12 points]}

Find the integer $a$ such that
\begin{qparts}
\item $a\equiv 58 \pmod{18}$ and $0 \leq a\leq 17$;
\item $a \equiv -142 \pmod{7}$ and $0 \leq a \leq 6$;
\item $a \equiv 17 \pmod{29}$ and $-14 \leq a \leq 14$;
\item $a \equiv -11 \pmod{21}$ and $110 \leq a \leq 130$.
\end{qparts}
Show your intermediate steps or briefly explain your process to justify your work.

\begin{solution}
    \begin{qparts}
        \item $a = 7$. The remainder of 58 divided by 17 is 7. 7 is between 0 and 17, so it is the answer.
        \item $a=2$. $-142 = 2 - 144$. Since 144 is a multiple of 6, 2 is congruent to this. 
        \item $a= -12$. $29 - 12 = 17$, so $17 \equiv -12$.
        \item $a = 115$. $6 \cdot 21 - 11 = 126 - 11 = 115$.
        \end{qparts}
\end{solution}
\clearpage
\subsection*{\probnum Multiple Modular Madness [14 points]}
For each of the questions below, answer ``always," ``sometimes," or ``never," then explain your answer. Your explanation should justify why you chose the answer you did, but does not have to be a rigorous proof.

\textbf{Hint:} Recall that if $a\equiv b\pmod{m}$ then there exists an integer $k$ such that $a=b+mk.$
\begin{qparts}
    \item Suppose $a \equiv 2 \pmod{21}$. When is $a \equiv 2 \pmod{7}$?
    \item Suppose $b \equiv 2 \pmod{7}$. When is $b \equiv 2 \pmod{21}$?
    \item Suppose $c \equiv 5 \pmod{8}$. When is $c \equiv 4 \pmod{16}$?
    \item Suppose $d \equiv 3 \pmod{21}$. When is $d \equiv 0 \pmod{6}$?
\end{qparts}

\begin{solution}
    \begin{qparts}
        \item Always. Since $a = 2 + 21k$, then $a = 2 + 7 \cdot 3k$. So $a \equiv 2 $ (mod 7).
        \item Sometimes. $a = 2 + 7k$. If $k$ is divisible by 3, then $7k$ is a multiple of 21, so the statement will be true. Otherwise, it is not true.
        \item Never. If $c = 4 + 16m$, then it is divisible by 4. But $c = 5 + 8k$ is not divisible by 4, so it would be a contradiction if this were true.
        \item Sometimes. Let $d = 3 + 21k$. If $k$ is even, then $d = 3 + 21 + 6 \cdot \frac{7k}{2} = 6(4 + \frac{7k}{2})$, which is divisible by 6. If not, then it is not true.
    \end{qparts}
\end{solution}
\clearpage
\subsection*{\probnum How Low Can You Go? [12 points]}
Suppose $a\equiv 3\pmod{10}$ and $b\equiv 8\pmod{10}$. In each part, find $c$ such that $0\leq c\leq 9$ and
\begin{qparts}
    \item $c \equiv 14a^2 - b^3 \pmod{10}$
    \item $c \equiv b^{15} - 99 \pmod{10}$
    \item $c\equiv a^{97} \pmod{10}$
\end{qparts}
Show your work! You should be doing the arithmetic/making substitutions \textbf{without using a calculator}. Your work must \textbf{not} include numbers above 100.

\begin{solution}
    \begin{qparts}
        \item
        \begin{align*}
            c &\equiv 14a^2 - b^3 \pmod{10} \\
            c &\equiv 4a^2 - b^3 \pmod{10} \\
            c &\equiv 4\cdot 3^2 - 8^3 \pmod{10} \\
            c &\equiv 4\cdot 9 - (-2)^3 \pmod{10} \\
            c &\equiv 4\cdot (-1) - (-8) \pmod{10} \\
            c &\equiv -12 \pmod{10} \\
            c &\equiv 8 \pmod{10} \\
        \end{align*}
        \item \begin{align*}
            c &\equiv b^{15} - 99 \pmod{10} \\
            c &\equiv 8^{15} - (-1) \pmod{10} \\
            c &\equiv \left((-2)^{3}\right)^5 + 1 \pmod{10} \\
            c &\equiv (-8)^5 + 1 \pmod{10} \\
            c &\equiv (2)^5 + 1 \pmod{10} \\
            c &\equiv 32 + 1 \pmod{10} \\
            c &\equiv 33 \pmod{10} \\
            c &\equiv 3 \pmod{10} \\
        \end{align*}
        \item \begin{align*}
            c &\equiv a^{97} \pmod{10} \\
            c &\equiv a^{95} \cdot a^2 \pmod{10} \\
            c &\equiv \left(a^5\right)^{19} \cdot a^2 \pmod{10} \\
            c &\equiv \left(8^5\right)^{19} \cdot 8^2 \pmod{10} \\
            c &\equiv \left((-2)^5\right)^{19} \cdot (-2)^2 \pmod{10} \\
            c &\equiv \left(32\right)^{19} \cdot 4 \pmod{10} \\
            c &\equiv \left(-2\right)^{19} \cdot 4 \pmod{10} \\
            c &\equiv \left((-2)^{5}\right)^3 \cdot (-2)^4 \cdot 4 \pmod{10} \\
            c &\equiv (-2)^3 \cdot (-2)^4 \cdot 4 \pmod{10} \\
            c &\equiv (-8) \cdot 16 \cdot 4 \pmod{10} \\
            c &\equiv 2 \cdot 6 \cdot 4 \pmod{10} \\
            c &\equiv 48 \pmod{10} \\
            c &\equiv 8 \pmod{10} \\
        \end{align*}
    \end{qparts}
\end{solution}

\subsection*{\probnum Be There or Be Square [16 points]}

Prove that if $n$ is an odd integer, then $n^2 \equiv 1 \pmod 8.$

\textbf{Note:} You \textbf{cannot} use the fact that all integers are equivalent to one of 0-7 (mod 8) without proof.

\begin{solution}
    If $n$ is odd, then it can be expressed as $n = 2m + 1$, where $m$ is an integer. Then $n^2 = (2m + 1)^2 = 4m^2 + 4m + 1 = 4(m^2 + m) + 1$. We know by previous homeworks that an integer has the same parity as its square, and the sum of two numbers with the same parity is even. So $m^2 + m$ is even, and can be expressed $m^2 + m = 2k$, where $k$ is an integer. So $n^2 = 8k + 1$, and its remainder when divided by 8 is 1. Thus $n^2 \equiv 1 \pmod{8}$.
\end{solution}

\clearpage
\subsection*{\probnum Functions and Fakers [16 points]}
Determine if each of the examples below are functions or not.
\begin{itemize}
    \item If it is not a function, explain why not.
    \item If it is a function, state whether or not it is bijective, and briefly justify your answer.
\end{itemize}
All domains and codomains are given as intended.

\begin{qparts}
    \item $f \colon \mathbb{R}^{\times} \to \mathbb{R}^{\times}$ such that $f(x) = x^{-1}.$

    \textbf{Note:} The set $\mathbb{R}^{\times}$ is the set $\mathbb{R}-\{0\}.$ Additionally, recall that $x^{-1}=\frac 1x.$
    
    \item $g \colon \mathbb{R} \to \mathbb{R}$ such that $g(x) = y$ iff $y \leq x.$
    \item $h \colon \textbf{U-M Courses} \to \{ \text{EECS}, \text{MATH} \}$ which maps each class to its department.
    \item $k \colon \textbf{U-M Courses} \to \mathbb{N}$ which maps each class to its course number
\end{qparts}
For example, $h(\text{EECS 203}) = \text{EECS}$ and $k(\text{EECS 203}) = 203.$

\textbf{Note:} For the purpose of parts (c) and (d), two courses are considered ``equal" if and only if they have the same department and course number. In particular, cross-listed courses are treated as distinct elements of $\textbf{U-M Courses}.$

\begin{solution}
    \begin{qparts}
        \item Yes, the multiplicative inverse is defined for all real numbers other than 0. The source of this function is exactly that set, so the map is defined for the whole domain. Additionally, every nonzero real number has a unique multiplicative inverse. So, this function is bijective.
        \item No, this results in more than 1 output for each input. For instance, take $x=1$. Then $y = 0$ and $y = 0.5$ both satisfy the map.
        \item No, U-M courses which are not EECS or MATH classes will not have a defined output. So, the map is not defined for the whole domain, meaning it is not a function.
        \item Yes, this is a function, assuming there don't exist any classes with multiple course numbers. However, this would not be bijective since not all elements of $\naturals$ are outputs of the function. For instance, 0 is not an output, since (as far as I know) there is no class with course number 0.
    \end{qparts}
\end{solution}

\clearpage
\subsection*{\probnum Fantastic Functions [18 points]}
For each of the functions below, determine whether it is (i) one-to-one, (ii) onto. Prove your answers.
\begin{qparts}
    \item $f \colon \mathbb{R} \to \mathbb R - \mathbb R^-,\ f(x) = e^{2x + 1}.$

    \item $g \colon \mathbb{R} - \left\{-\frac{2}{5}\right\} \to \mathbb{R} - \left\{\frac{3}{5}\right\},\ g(x) = \frac{3x - 1}{5x + 2}.$

    \item$h \colon \mathbb{Z} \times \mathbb {Z} \to \mathbb{Z},\ h(m, n) = |m|-|n|$
\end{qparts}


\begin{solution}
    \begin{qparts}
        \item (i) Yes. Let $f(x) = f(y) = z$ for $x, y \in \reals$, $z \in \mathbb R - \mathbb R^-$. Then \begin{align*}
            z &= e^{2x+1} \\
            x &= \frac{\ln z - 1}{2} \\
            z &= e^{2y+1} \\
            y &= \frac{\ln z - 1}{2} \\
            x = y
        \end{align*}
        Since $x = y$, $f$ is injective.
        \par (ii) No. $0 \in \mathbb R - \mathbb R^-$ is not a possible output for $e^x$, so it is impossible for it to be an output for $f$.
    
        \item (i) Yes. Let $g(x) = g(y) = z$ for $x, y \in \mathbb{R} - \left\{-\frac{2}{5}\right\}$, $z \in \mathbb R - \left\{\frac{3}{5}\right\}$. Then \begin{align*}
            z &= \frac{3x - 1}{5x + 2} \\
            x &= \frac{2z+1}{3-5z} \\
            z &= \frac{3y - 1}{5y + 2} \\
            y &= \frac{2z+1}{3-5z} \\
            x = y
        \end{align*}
        Since $x = y$, $g$ is injective.
        \par (ii) Yes. For any $y \in \reals - \left\{\frac{3}{5}\right\}$, a there exists $x = \frac{2y+1}{3-5y}$ such that $g(x) = y$. Thus, the function is surjective.
    
        \item (i) No, this function is not injective. For instance, $1 = h(2, 1) = h(3, 2)$.
        \par (ii) Yes, this function is onto. For any $y \in \ints$, there exists $m = |n| + y$, $n \in \ints$ which satisfy $h(m, n) = y$.
    \end{qparts}
\end{solution}

\clearpage
\subsection*{\probnum Comp$\circ$sition$($Functions$)$ [12 points]}
For each of the following pairs of functions $f$ and $g$, find $f \circ g$ and $g \circ f$. Make sure to include the domain and codomain of each composed function you give. If either can't be computed, explain why.
\begin{qparts}
    \item $f\colon \mathbb{N} \to \mathbb{Z}^{+},\; f(x) = x + 1$
    
    $g\colon\mathbb{Z}^{+}\to\mathbb{N},\; g(x) = x^2 - 1$
    
    \item $f\colon\mathbb{Z}\to\mathbb{R},\; f(x) = \left(\frac{3}{2} x + 3\right)^{3}$
    
    $g\colon\mathbb{R} \rightarrow \mathbb{R}_{\geq 0},\; g(x) = |x|$

    \textbf{Note:} $\mathbb{R}_{\geq 0}$ is the set of real numbers greater than or equal to 0.
\end{qparts}


\begin{solution}
    \begin{qparts}
        \item $f \circ g$ and $g \circ f$ are both defined because the domains and codomains match. $f(g(x)) = (x^2 - 1) + 1 = x^2$, and $g(f(x)) = (x+ 1)^2 - 1 = x^2 + 2x$. So 
        \[f \circ g : \ints^+ \to \ints^+, (f \circ g)(x) = x^2 \]
        \[g \circ f : \naturals \to \naturals, (g \circ f)(x) = x^2 + 2x\]
        
        \item The codomain of $f$ is the same as $g$'s domain, but not the other way around. So $g \circ f$ will be defined, but $f \circ g$ will not be. $g(f(x)) = \left|(\frac{3}{2}x + 3)^3\right|$.
        \[f \circ g : \ints \to \reals_{\geq 0}, g(f(x)) = \left|\frac{3}{2}x + 3\right|^3\]
    \end{qparts}
\end{solution}

\end{document}