\documentclass[12pt]{exam}

% essential packages
\usepackage{fullpage} % margin formatting
\usepackage{enumitem} % configure enumerate and itemize
\usepackage{amsmath, amsfonts, amssymb, mathtools} % math symbols
\usepackage{xcolor, colortbl} % colors, including in tables
\usepackage{makecell} % thicker \Xhline in table
\usepackage{graphicx} % images, resizing

% sometimes needed packages
\usepackage{hyperref} % hyperlinks
% \hypersetup{colorlinks=true, urlcolor=blue}
% \usepackage{logicproof} % natural deduction
% \usepackage{tikz} % drawing graphs
% \usetikzlibrary{positioning}
% \usepackage{multicol}
% \usepackage{algpseudocode} % pseudocode

% paragraph formatting
\setlength{\parskip}{6pt}
\setlength{\parindent}{0cm}

% newline after Solution:
\renewcommand{\solutiontitle}{\noindent\textbf{Solution:}\par\noindent}

% less space before itemize/enumerate
\setlist{topsep=0pt}

% creates \filcl to grey out cells for groupwork grading
\newcommand{\filcl}{\cellcolor{gray!25}}

% creates \probnum to get the problem number
\newcounter{probnumcount}
\setcounter{probnumcount}{1}
\newcommand{\probnum}{\arabic{probnumcount}. \addtocounter{probnumcount}{1}}

% use roman numerals by default
\setlist[enumerate]{label={(\roman*)}}

% creates custom list environments for grading guidelines, question parts
\newlist{guidelines}{itemize}{1}
\setlist[guidelines]{label={}, left=0pt .. \parindent, nosep}
\newlist{gwguidelines}{enumerate}{1}
\setlist[gwguidelines]{label={(\roman*)}, nosep}
\newlist{qparts}{enumerate}{2}
\setlist[qparts]{label={(\alph*)}}
\newlist{qsubparts}{enumerate}{2}
\setlist[qsubparts]{label={(\roman*)}}
\newlist{stmts}{enumerate}{1}
\setlist[stmts]{label={(\roman*)}, nosep}
\newlist{pflist}{itemize}{4}
\setlist[pflist]{label={$\bullet$}, nosep}
\newlist{enumpflist}{enumerate}{4}
\setlist[enumpflist]{label={(\arabic*)}, nosep}

\printanswers

\newcommand{\prevhwnum}{8}
\newcommand{\hwnum}{9}

\begin{document}
%%%%%%%%%%%%%%% TITLE PAGE %%%%%%%%%%%%%%%
\title{EECS 203: Discrete Mathematics\\
  Winter 2024\\
  Homework \hwnum{}}
\date{}
\author{}
\maketitle
\vspace{-50pt}
\begin{center}
  \huge Due \textbf{Thursday, April. 11th}, 10:00 pm\\
\Large No late homework accepted past midnight.\\
\vspace{10pt}
\large Number of Problems: $8+2$
\hspace{3cm}
Total Points: $100+30$
\end{center}
\vspace{25pt}
\begin{itemize}
    \item \textbf{Match your pages!} Your submission time is when you upload the file, so the time you take to match pages doesn't count against you.
    \item Submit this assignment (and any regrade requests later) on Gradescope. 
    \item Justify your answers and show your work (unless a question says otherwise).
    \item By submitting this homework, you agree that you are in compliance with the Engineering Honor Code and the Course Policies for 203, and that you are submitting your own work.
    \item Check the syllabus for full details.
\end{itemize}
\newpage
%%%%%%%%%%%%%%% TITLE PAGE %%%%%%%%%%%%%%% 

\section*{Individual Portion}

\subsection*{\probnum How Many pa55words? [12 points]}
A password consists of exactly 6 characters, where each character is either a lowercase letter (a-z) or a digit (0-9). However, the password must contain at least 2 digits and at least 2 lowercase letters. How many different passwords are possible?

\begin{solution}
    2 digits, 4 letters:
    $C(6, 2)$ for choosing the places of the digits. \\
    $26^4$ for number of ways to choose the 4 lowercase letters. \\
    $10^2$ for the number of ways to choose the 2 digits.\\
    \par 3 digits, 3 letters:
    $C(6, 3)$ for choosing the places of the digits. \\
    $26^3$ for number of ways to choose the 3 lowercase letters. \\
    $10^3$ for number of ways to choose the 3 digits. \\
    \par 4 digits, 2 letters:
    $C(6, 4)$ for choosing the places of the digits. \\
    $26^2$ for number of ways to choose the 2 lowercase letters. \\
    $10^4$ for number of ways to choose the 4 digits. \\
    \par In total: 
    \[ {6 \choose 2} \cdot 26^4 \cdot 10^2 + {6 \choose 3} \cdot 26^3 \cdot 10^3 + {6 \choose 4} \cdot 26^2 \cdot 10^4 = \boxed{1.138384\times 10^9} \]
\end{solution}


\subsection*{\probnum (Not So) Round and Round [12 points]}
Suppose we have a square-shaped table which seats 3 people on each side. How many ways are there to seat 12 people at the table where seatings are considered the same if everyone is in the same group of 3 on a side?

\begin{solution}
    This will be equivalent to the permutations of 12 people divided by the ways to rearrange each group of 3 on a side. The number of rearrangements is $3!$ for each side, so $3!^4$ in total.
    \[ \frac{12!}{3!^4} = \boxed{369600} \]
\end{solution}


\subsection*{\probnum My backpack is too heavy [12 points]}
How many ways are there to distribute seven textbooks into eleven backpacks if \textbf{each backpack must have at most one textbook in it} and
\begin{qparts}
    \item all of the textbooks and all of the backpacks are unique?
    \item all of the textbooks are unique, but all of the backpacks are identical?
    \item all of the textbooks are identical, but the backpacks are all unique?
    \item all of the textbooks and all of the backpacks are identical?
\end{qparts}

\begin{solution}
    \begin{qparts}
        \item For each book, there is a choice for which backpack it is placed in. The order of the empty backpacks does not matter. Also, the books can only be placed into one backpack. We do not need to account for rearrangements of the books after calculating the permutations of 7 backpacks from 11 total, since this will account for the permutations of books within backpacks as well. So this is $_{11}P_7 = \boxed{1663200}$
        \item There will only be one way to do this. Each backpack will only have one book at most, but all books must be placed into a backpack. Since the order of the filled backpacks and empty backpacks do not matter since they are indistinguishable, there is only $\boxed{\text{one}}$ possible way to distribute the books 7 filled backpacks, and 4 empty ones.
        \item Each backpack can only be filled or empty, and there is no other way in which order matters. So the number of ways this can happen is the ways to choose 7 backpacks to be filled from the 11. This is $C(11, 2) = \boxed{55}$.
        \item Since each filled backpack can only have 1 book, there is no way to distinguish the filled bags from each other. The empty bags are identical to each other. Also, all the books must be distributed. So there is only 1 way to distribute the books: 7 filled backpacks, and 4 empty ones.
    \end{qparts}
\end{solution}


\subsection*{\probnum Seeressously? [12 points]}
How many strings with six or more characters can be formed from the letters in ``SEERESS"? \textbf{Simplify your answer.} You may use a calculator to help you simplify.

\begin{solution}
    Strings of 7 characters: \\
    There are 7 characters but 2 subsets of 3 identical letters.\\
    \[\dfrac{7!}{3! 3!} = 140\]
    Strings of 6 characters: \\
    We use cases to omit 1 letter each.\\
    Omit 1 S:
    \[ \dfrac{6!}{3!2!} \] 
    Omit 1 E:
    \[ \dfrac{6!}{3!2!} \] 
    Omit 1 R:
    \[ \dfrac{6!}{3!3!} \] 
    Total: 
    \[ \frac{7!}{3!3!}+\frac{6!}{3!3!}+\frac{6!}{3!2!}+\frac{6!}{3!2!} = 280 \]
\end{solution}


\subsection*{\probnum Jackpot! [12 points]}
Suppose there is a lottery where the organizers pick a set of 11 distinct numbers. A player then picks 7 distinct numbers and wins when all 7 are in the set chosen by the organizers. Numbers chosen by both the players and organizers come from the set $\{ 1, 2, ..., 80 \}$.

\begin{qparts}
    \item Let the sample space, $S$, be all the sets of 7 numbers the player can choose. What is $|S|$?
    \item Let $E$ be the event that all the numbers the player chooses are in the winning set. What is $|E|$?
    \item What is the probability of winning? As a reminder, you may leave your answer unsimplified.
    \item Alternatively we can choose a different sample space $S'$ which is all the sets of 11 numbers that the organizers can choose. What is $|S'|$? What is the event $E'$ that the player has a winning set, and what is $|E'|$?
    \item What is the probability of winning given your answer to (d)? Use a calculator to verify this is the same as your answer to (c).
\end{qparts}

\begin{solution}
    \begin{qparts}
        \item The cardinality of the sample space is the number of ways to choose 7 numbers from 80 total numbers. This is $ {80 \choose 7} = \boxed{3.1767164\times10^{9}}$.
        \item We can choose an arbitrary winning set of numbers wlog. So, we choose the numbers from 1 to 11 inclusive. The number of ways 7 numbers could be chosen of this set is $ {11 \choose 7} = \boxed{330} $.
        \item This is the cardinality of the event divided by the size of the sample space. 
        \[ \dfrac{{11 \choose 7}}{{80 \choose 7}} = \boxed{\frac{3}{28879240}} \]
        \item $|S'| = {80 \choose 11}$ (choose 11 numbers from 80) and $|E'| = {73 \choose 4}$ (7 numbers are in the player's chosen set, so 4 of the organizers' can be chosen from the other numbers).
        \item \[ \dfrac{{73 \choose 4}}{{80 \choose 11}} = \boxed{\frac{3}{28879240}} = \dfrac{{11 \choose 7}}{{80 \choose 7}} \]
    \end{qparts}
\end{solution}


\subsection*{\probnum Rollin' in the Deep [12 points]}
You roll three, fair, 6-sided dice. What is the probability that the sum of the dice is less than 17?

\begin{solution}
    We use complementary counting. We assume the dice are distinguishable. There is only 1 way that the dice sum to 18, since they must all be 6. And there are 3 ways that the dice can sum to 17, since one of the dice is 5 and the rest are 6. So there are 4 total ways that the dice can sum to 17 or greater. In total, there are $6^3 = 216$ ways for the dice to be rolled. So the probability that the dice sum to less than 17 is; 
    \[ 1 - \frac{4}{216} = \boxed{\frac{53}{54}} = 0.9814814815 \]
\end{solution}


\subsection*{\probnum Addition Condition [12 points]}
A pair of six-sided dice is rolled, but you do not see the results. You ask your friend whether at least one die came up six, and they respond yes.

What is the probability that the sum of the numbers that came up on the two dice is seven, given the information provided by your friend?

\begin{solution}
    We use Bayes' theorem, and also assume the dice are distinguishable. The ways that at least one die can come up 6 is $6 + 6 - 1 = 11$ (by inclusion-exclusion for each die coming up 6). The ways in which the numbers can sum to 7 with at least one 6 is 2. So the total probability is $\boxed{\dfrac{2}{11} = 0.1818181818}$.
\end{solution}


\subsection*{\probnum Conditional probability \& Independence [16 points]}

Suppose you put two dice in a bag: one of the dice has a 6 on every face, and the other is a standard 6-sided die. You choose one die at random, roll it, and get a 6.

\begin{qparts}
    \item If you roll the same die, what is the probability that the next roll is also a 6?
    \item If you roll the same die both times, are rolling a 6 on the first roll and rolling a 6 on the second roll independent events?
\end{qparts}

\begin{solution}
    \begin{qparts}
        \item The probability is $\dfrac{1}{6}$.
        \item Yes. Their joint probability is $\dfrac{1}{36}$, and their individual properties are each $\dfrac{1}{6}$. Since $\dfrac{1}{36} = \left(\dfrac{1}{6}\right)^2$, their joint probability is equal to the product of their individual probabilities, and they are independent.
    \end{qparts}
\end{solution}

\end{document}