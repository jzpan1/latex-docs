\documentclass[12pt]{exam}

% essential packages
\usepackage{fullpage} % margin formatting
\usepackage{enumitem} % configure enumerate and itemize
\usepackage{amsmath, amsfonts, amssymb, mathtools} % math symbols
\usepackage{xcolor, colortbl} % colors, including in tables
\usepackage{makecell} % thicker \Xhline in table
\usepackage{graphicx} % images, resizing

% sometimes needed packages
\usepackage{hyperref} % hyperlinks
% \hypersetup{colorlinks=true, urlcolor=blue}
% \usepackage{logicproof} % natural deduction
% \usepackage{tikz} % drawing graphs
% \usetikzlibrary{positioning}
% \usepackage{multicol}
% \usepackage{algpseudocode} % pseudocode

% paragraph formatting
\setlength{\parskip}{6pt}
\setlength{\parindent}{0cm}
\newcommand{\reals}{\mathbb{R}}
\newcommand{\naturals}{\mathbb{N}}
\newcommand{\ints}{\mathbb{Z}}
\newcommand{\transpose}{^\top}

% newline after Solution:
\renewcommand{\solutiontitle}{\noindent\textbf{Solution:}\par\noindent}

% less space before itemize/enumerate
\setlist{topsep=0pt}

% creates \filcl to grey out cells for groupwork grading
\newcommand{\filcl}{\cellcolor{gray!25}}

% creates \probnum to get the problem number
\newcounter{probnumcount}
\setcounter{probnumcount}{1}
\newcommand{\probnum}{\arabic{probnumcount}. \addtocounter{probnumcount}{1}}

% use roman numerals by default
\setlist[enumerate]{label={(\roman*)}}

% creates custom list environments for grading guidelines, question parts
\newlist{guidelines}{itemize}{1}
\setlist[guidelines]{label={}, left=0pt .. \parindent, nosep}
\newlist{gwguidelines}{enumerate}{1}
\setlist[gwguidelines]{label={(\roman*)}, nosep}
\newlist{qparts}{enumerate}{2}
\setlist[qparts]{label={(\alph*)}}
\newlist{qsubparts}{enumerate}{2}
\setlist[qsubparts]{label={(\roman*)}}
\newlist{stmts}{enumerate}{1}
\setlist[stmts]{label={(\roman*)}, nosep}
\newlist{pflist}{itemize}{4}
\setlist[pflist]{label={$\bullet$}, nosep}
\newlist{enumpflist}{enumerate}{4}
\setlist[enumpflist]{label={(\arabic*)}, nosep}

\printanswers

\newcommand{\prevhwnum}{4}
\newcommand{\hwnum}{5}

\begin{document}
%%%%%%%%%%%%%%% TITLE PAGE %%%%%%%%%%%%%%%
\title{EECS 203: Discrete Mathematics\\
  Winter 2024\\
  Homework \hwnum{}}
\date{}
\author{}
\maketitle
\vspace{-50pt}
\begin{center}
  \huge Due \textbf{Thursday, Mar. 7th}, 10:00 pm\\
\Large No late homework accepted past midnight.\\
\vspace{10pt}
\large Number of Problems: $7+2$
\hspace{3cm}
Total Points: $100+30$
\end{center}
\vspace{25pt}
\begin{itemize}
    \item \textbf{Match your pages!} Your submission time is when you upload the file, so the time you take to match pages doesn't count against you.
    \item Submit this assignment (and any regrade requests later) on Gradescope. 
    \item Justify your answers and show your work (unless a question says otherwise).
    \item By submitting this homework, you agree that you are in compliance with the Engineering Honor Code and the Course Policies for 203, and that you are submitting your own work.
    \item Check the syllabus for full details.
\end{itemize}
\newpage
%%%%%%%%%%%%%%% TITLE PAGE %%%%%%%%%%%%%%% 

\section*{Individual Portion}

\subsection*{\probnum Easy as 3, 18, 93 [16 points]}

 Let $P(n)$ be the statement that $3 + 3 \cdot 5 + 3 \cdot 5^2 + ... + 3 \cdot 5^n = \frac{3 (5^{n+1} - 1)}{4}$. In this problem, we will prove using weak induction that $P(n)$ is true whenever $n$ is a non-negative integer.
\begin{parts}
    \item What is the statement $P(0)$? Complete the base case by showing that $P(0)$ is true.
    \item In the base case we prove $P(0)$; what do you need to prove in the inductive step?
    \item What is the inductive hypothesis for your proof?
    \item Complete the inductive step, indicating where you used the inductive hypothesis.
    
    \textit{Reminder:} You should prove this equation using a chain of equalities, starting on one side and transforming it into the other side.  You should \textbf{not} start with the equation you want to prove and transform both sides to be equal.
    \item Explain why this proof shows $P(n)$ is true for all non-negative integers $n$.
\end{parts}

\begin{solution}
    \begin{parts}
        \item $P(0)$ is the statement that $3 = \frac{3(5-1)}{4}$. This is true since $\frac{3(5-1)}{4} = \frac{12}{4} = 3$.
        \item In the inductive step, you should prove that $P(n)$ implies $P(n+1)$ for any nonnegative integer $n$.
        \item Assume that $P(n)$ is true for some $n \geq 0 \in \ints$; that is, $3 + 3 \cdot 5 + 3 \cdot 5^2 + ... + 3 \cdot 5^n = \frac{3 (5^{n+1} - 1)}{4}$ is true.
        \item By the inductive hypothesis, $3 + 3 \cdot 5 + 3 \cdot 5^2 + ... + 3 \cdot 5^n = \frac{3 (5^{n+1} - 1)}{4}$. Adding $3 \cdot 5^{n+1}$ to both sides, 
        \begin{align*}
            3 + 3 \cdot 5 + 3 \cdot 5^2 + ... + 3 \cdot 5^n + 3 \cdot 5^{n+1} &= \frac{3 (5^{n+1} - 1)}{4} + 3 \cdot 5^{n+1} \\
            &=\frac{3 (5^{n+1} - 1) + 4 \cdot 3 \cdot 5^{n+1}}{4} \\
            &=\frac{3 ( 5 \cdot 5^{n+1}) - 3}{4} \\
            3 + 3 \cdot 5 + 3 \cdot 5^2 + ... + 3 \cdot 5^n + 3 \cdot 5^{n+1} &=\frac{3 ( 5^{n+2} - 1)}{4}
        \end{align*}
        So $P(n)$ implies $P(n+1)$. 
        \item Since $P(0)$ holds, $P(n)$ holds for all nonnegative integers $n$ by induction.
    \end{parts}
\end{solution}


\subsection*{\probnum Inequality Induction [16 points]}
 Let $P(n)$ be the following inequality: \(2^n > n\). Use weak induction to prove that $P(n)$ is true for all positive integers.
\begin{parts}
    \item What is the statement $P(1)$? Complete the base case by showing that $P(1)$ is true. 
    \item What do you want to show in the inductive step?
    \item What is the inductive hypothesis for your proof?
    \item Complete the inductive step, indicating where you used the inductive hypothesis.
    \item Conclude your proof by explaining why the above shows $P(n)$ is true for all positive integers $n$.
\end{parts}
\begin{solution}
    \begin{parts}
        \item $P(1)$ is the statement that $2^1 > 1$. This is true since $2^1 = 2 > 1$.
        \item In the inductive step, you should prove that $P(n)$ implies $P(n+1)$ for any positive integer $n$.
        \item Assume that $P(n)$ is true; that is, for some positive integer $n$, $2^n > n$ is true.
        \item By the inductive hypothesis, we know $2^n > n$. Multiplying by 2, we find $2^{n + 1} > 2n$. Since $n$ is a positive integer, $2n \geq n + 1$. So $2^{n + 1} > n + 1$.
        \item We have shown that $P(n)$ implies $P(n+1)$. Since $P(1)$ is true, $P(n)$ is true for integer $n > 0$.
    \end{parts}
\end{solution}

\subsection*{\probnum Divisible Induction [16 points]}
Prove by induction that 5 divides $3^{4n}+4$ whenever $n$ is a positive integer.

\begin{solution}
    Let $P(n)$ be the statement that 5 divides $3^{4n}+4$ for some positive integer $n$. 
    \par Base case: $P(1)$: 5 divides $3^{4\cdot 1} + 4 = 81 + 4 = 85$. This statement is true.
    \par Inductive step: Assume $P(k)$ is true for some positive integer $k$. Then we know that $3^{4\cdot k} + 4 = 5m$ for some integer $m$. Multiplying both sides by $3^4$, $3^{4 \cdot (k+1)} + 3^4 \cdot 4 = 3^4 \cdot 5m$. Then 
    \begin{align*}
        3^{4 \cdot (k+1)} + 3^4 \cdot 4 &= 3^4 \cdot 5m \\
        3^{4 \cdot (k+1)} + 4 &=  3^4 \cdot (5m - 4) + 4\\
        &= 405m - 324 + 4
        &= 405m - 320 \\
        &= 5(81m - 64)
    \end{align*}
    Since $(81m - 64)$ is an integer, $3^{4 \cdot (k+1)} + 4$ is divisible by 5. So we have shown that $P(k)$ implies $P(k+1)$.
    Since $P(1)$ is true, $P(n)$ is true for any positive integer $n$.
\end{solution}


\subsection*{\probnum Please Pretend Postage Pun Present [12 points]}
Let $P(n)$ be the predicate ``$n$ cents can be formed using $3$ and $7$ cent stamps."
\begin{qparts}
    \item Find the smallest $c\in \mathbb N$ so that $\forall n\ge c,\ P(n)$.
    \item Prove by induction that $\forall n\ge c,\ P(n)$. Use the minimum number of base cases needed.
\end{qparts}

\begin{solution}
    \begin{qparts}
        \item $c = 12$.
        \item Base cases: 
        \begin{align*}
            P(12): 12 &= 4 \cdot 3 + 0 \cdot 7 \\
            P(13): 13 &= 2 \cdot 3 + 1 \cdot 7 \\
            P(14): 14 &= 0 \cdot 3 + 2 \cdot 7 \\
        \end{align*} 
        \par Inductive step: Assume $P(k-3)$ is true, and $(k-3)$ cents can be formed by $n$ multiples of 3 cents and $m$ multiples of 7 cent stamps, where $n, m$ are nonnegative integers. Then $P(k)$ is true, since $k = (k - 3) + 3 = n \cdot 3 + m \cdot 7 + 3 = (n + 1)\cdot 3 + m \cdot 7$. Since $P(12), P(13), P(14)$ are all true, the $P(n)$ is true for any integer $n \geq 12$.
    \end{qparts}
\end{solution}


\subsection*{\probnum Inductive Delights [14 points]}
Assume that a chocolate bar consists of $n\geq 1$ squares arranged in a rectangular pattern. Any rectangular piece of the bar including the entire bar can be broken along a vertical or a horizontal line separating the squares. Assuming you can only break the bar along one axis at a time, determine how many breaks you must successively make to break the bar into $n$ separate squares. Use \textbf{strong induction} to prove your answer.

\begin{solution}
    Let $B(n)$ be the minimum number of breaks required to split a chocolate bar of $n$ squares into $n$ separate squares. We will show that $B(n) = n - 1$.
    \par Base cases: $B(1) = 0 = 1-1$
    \par Inductive hypothesis: Assume $B(j) = j-1$ for all integers $j$ such that $1 \leq j < k$, $k \in \ints^+$. 

    \par Inductive step: Breaking a chocolate bar of $k\in \ints^+$ squares will result in two pieces, with $j_0$ and $k - j_0$ pieces respectively. Note that $j_0$ and $k - j_0$ are strictly less than $k$. Then by the inductive hypothesis, the remaining number of breaks to split the pieces into $k$ squares is $B(j_0) + B(k- j_0) = j_0 - j_9 + k_0 - 2$. So the total number of breaks is $1 + k - 2 = k - 1$. Thus we have shown that if $B(j) = j - 1$ is true for all integers $j$ satisfying $1 \leq j < k$ for $k \in \ints^+$, it implies that $B(k) = k - 1$. 
    \par Since $B(1) = 1 - 1 = 0$, $B(n) = n - 1$ for all positive integers $n$ by induction. 
\end{solution}


\subsection*{\probnum A Mess of Messages [12 points]} 
We are sending messages made up of the characters ``a'', ``b'', and ``c''. An ``a'' takes 1 microsecond to send, and a ``b'' or ``c'' takes 2 microseconds to send. Let $M(n)$ denote the number of distinct messages we can send using exactly $n$ microseconds (in particular, the message cannot be sent in fewer than $n$ microseconds), for $n \ge 0$.

\begin{qparts}
    \item Give a recurrence relation for $M(n).$
    \item Give the initial conditions for your recurrence. Include only the minimum necessary conditions.
\end{qparts}

\begin{solution}
    \begin{qparts}
        \item There are 2 ways to add onto a $n-2$ microsecond message to become a $n$ microsecond message: either add a ``b" or a ``c". To add onto a $n-1$ microsecond message to become a $n$ microsecond message, only adding ``a'' is possible. Note that adding 2 ``a'''s to $n-2$ is already included in the $n-1$ case, so it is not a valid way to make $n$ from $n-2$. So $M(n) = M(n - 1) + 2M(n-2)$.
        \item $M(0) = 1$ \\ $M(1) = 1$
    \end{qparts}
\end{solution}

\subsection*{\probnum Carrot the Cat [14 points]}
Carrot the cat likes taking naps in one of four locations: the rug, the bed, the ledge, and the sink. Carrot has the following conditions:
\begin{itemize}
    \item He will not sleep in the sink twice in a row
    \item He will sleep on the ledge only if he slept on the rug the previous time
\end{itemize}
Let $L(n)$ be the number of possible sequences of locations for $n$ naps, where $n \ge 0.$

\begin{qparts}
    \item Give a recurrence relation for $L(n).$
    \item Give the initial conditions for your recurrence. Include only the minimum necessary conditions.
\end{qparts}

\begin{solution}
    The wording of the question is slightly ambiguous, but I assume that the ledge/rug nap relation is not if and only if; i.e. a ledge nap implies the last nap was a rug nap, but a rug nap does not imply the next nap will be a ledge nap.
    \begin{qparts}
        \item Let $r(n), b(n), l(n), s(n)$ represent the number of sequences ending in rug, bed, ledge, and sink naps resepctively for $n$ naps. Then $L(n) = r(n) + b(n) + l(n) + s(n)$. We then find formulas for these respectively. 
        \par Bed naps and rug naps are possible no matter what the previous nap location was. So $r(n) = b(n) = L(n-1)$.
        \par Ledge naps are only possible when the last nap was a rug nap. So $l(n) = r(n-1) = L(n-2)$.
        \par Sink naps are possible if the last nap was not a sink nap. So $s(n) = r(n - 1) + b(n - 1) + l(n - 1) = 2L(n-2) + L(n-3)$.
        \par In total, \begin{align*}
            L(n) &= r(n) + b(n) + l(n) + s(n) \\
            &=  L(n-1) + L(n-1) + L(n-2) + 2L(n-2) + L(n-3)
        \end{align*}
        \[ \boxed{L(n) = 2L(n-1) + 3L(n-2) + L(n-3)} \]
        \item \begin{align*}
            L(0) &= 1 \\
            L(1) &= 3 \\
            L(2) &= 9
        \end{align*}
    \end{qparts}
\end{solution}

\pagebreak

\end{document}